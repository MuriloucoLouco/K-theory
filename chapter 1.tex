\chapter{\scshape Vector Bundles}

\section{Basic definitions.}\label{sec:1.1} We shall develop the theory of \textit{complex} vector bundles only, though much of the elementary theory is the same for real and symplectic bundles. Therefore, by vector space, we shall always understand complex vector space unless otherwise specified. \par

Let $X$ be a topological space. A \textit{family of vector spaces over} $X$ is a topological space $E$, together with:
\begin{enumerate}[(i)]
    \item a continuous map $p: E \to X$
    \item a finite dimensional vector space structure on each
    \begin{equation*}
        E_x = p^{-1}(x) \qquad \text{for } x \in X \text{,}
    \end{equation*}
    compatible with the topology on $E_x$ induced from $E$.
\end{enumerate}

The map $p$ is called the projection map, the space $E$ is called the total space of the family, the space $X$ is called the base space of the family, and if $x \in X$, $E_x$ is called the fiber over $x$. \par

A \textit{section} of a family $p: E \to X$ is a continuous map $s: X \to E$ such that $p \circ s(x) = x$ for all $x \in X$. \par

\pagebreak
A \textit{homomorphism} from one family $p: E \to X$ to another family $q: F \to X$ is a continuous map $\varphi: E \to F$ such that:
\begin{enumerate}[(i)]
    \item $q \circ \varphi = p$
    \item for each $x \in X$, $\varphi: E_x \to F_x$ is a linear map of vector spaces.
\end{enumerate}

We say that $\varphi$ is an \textit{isomorphism} if $\varphi$ is bijective and $\varphi^{-1}$ is continuous. If there exists an isomorphism between $E$ and $F$, we say that they are isomorphic. \par

\textbf{Example 1.} Let $V$ be a vector space, and let $E = X \times V$, $p: E \to X$ be the projection onto the first factor. $E$ is called the \textit{product family} with fiber $V$. If $F$ is any family which is isomorphic to some product family, $F$ is said to be a \textit{trivial} family. \par

If $Y$ is a subspace of $X$, and if $E$ is a family of vector spaces over $X$ with projection $p$, $p: p^{-1}(Y) \to Y$ is clearly a family over $Y$. We call it the \textit{restriction} of $E$ to $Y$, and denote it by $E \vert Y$. More generally, if $Y$ is any space, and $f: Y \to X$ is a continuous map, then we define the induced family $f^*(p): f^*(E) \to Y$ as follows: \par

$f^*(E)$ is the subspace of $Y \times E$ consisting of all points $(y, e)$ such that $f(y) = p(e)$, together with the obvious projection maps and vector space structures on the fibers. If $g: Z \to Y$, then there is a natural isomorphism $g^* f^*(E) \cong (fg)^*(E)$ given by sending each point of the form $(z, e)$ into the point $(z, g(z), e)$, where $z \in Z$, $e \in E$. If $f: Y \to X$ is an inclusion map, clearly there is an isomorphism $E \vert Y \cong f^*(E)$ given by sending each $e \in E$ into the corresponding $(p(e), e)$.

A family $E$ of vector spaces over $X$ is said to be \textit{locally trivial} if every $x \in X$ posesses a neighborhood $U$ such that $E \vert U$ is trivial. A locally trivial family will also be called a \textit{vector bundle}. A trivial family will be called a trivial bundle. If $f: Y \to X$, and if $E$ is a vector bundle over $X$, it is easy to see that $f^*(E)$ is a vector bundle over $Y$. We shall call $f^*(E)$ the induced bundle in this case. \par \pagebreak

\textbf{Example 2.} Let $V$ be a vector space, and let $X$ be its associated projective space. We define $E \subset X \times V$ to be the set of all $(x, v)$ such that $x \in X$, $v \in V$, and $v$ lies in the line determining $x$. We leave it to the reader to show that $E$ is actually a vector bundle. \par

Notice that if $E$ is a vector bundle over $X$, then $\dim(E_x)$ is a locally constant function on $X$, and hence is a constant on each connected component of $X$. If $\dim(E_x)$ is a constant on the whole of $X$, then $E$ is said to have a dimension, and the dimension of $E$ is the common number $\dim(E_x)$ for all $x$. (Caution: the dimension of $E$ so defined is usually different from the dimension of $E$ as a topological space.) \par

Since a vector bundle is locally trivial, any section of a vector bundle is locally described by a vector valued function on the base space. If $E$ is a vector bundle, we denote by $\Gamma(E)$ the set of all sections of $E$. Since the set of functions on a space with values in a fixed vector space is itself a vector space, we see that $\Gamma(E)$ is a vector space in a natural way. \par 

Suppose that $V, W$ are vector spaces, and that $E = X \times V$, $F = X \times W$ are the corresponding product bundles. Then any homomorphism $\varphi: E \to F$ determines a map $\Phi: X \to \Hom(V, W)$ by the formula $\varphi(x, v) = (x,  \Phi(x)v)$. Moreover, if we give $\Hom(V, W)$ its usual topology, then $\Phi$ is continuous; conversely, any such continuous map $\Phi: X \to \Hom(V, W)$ determines a homomorphism $\varphi: E \to F$. (This is most easily seen by taking bases ${e_i}$ and ${f_i}$ for $V$ and $W$ respectively. Then each $\Phi(x)$ is represented by a matrix $\Phi(x)_{i,j}$, where
\begin{equation*}
    \Phi(x)e_i = \sum_j \Phi(x)_{i,j} f_j
\end{equation*}
The continuity of either $\varphi$ or $\Phi$ is equivalent to the continuity
of the functions $\Phi(x)_{i,j}$.) \par 

Let $\Iso(V, W) \subset \Hom(V, W)$ be the subspace of all isomorphisms between $V$ and $W$. Clearly, $\Iso(V, W)$ is an open set in $\Hom(V, W)$. Further, the inverse map $T \to T^{-1}$ gives us a continuous map $\Iso(V, W) \to \Iso(W, V)$. Suppose that $\varphi: E \to F$ is such that $\varphi_x: E_x \to F_x$ is an isomorphism for all $x \in X$. This is equivalent to the statement that $\Phi(x) \subset \Iso(V, W)$. The map $x \to \Phi(x)^{-1}$ defines $\Psi: X \to \Iso(W, V)$, which is continuous. Thus the corresponding map $\psi: F \to E$ is continuous. Thus $\varphi: E \to F$ is an isomorphism if and only if it is bijective or, equivalently, $\varphi$ is an isomorphism if and only if each $\varphi_x$ is an isomorphism. Further, since $\Iso(V, W)$ is open in $\Hom(V, W)$, we see that for any homomorphism $\varphi$, the set of those points $x \in X$ for which $\varphi_x$ is an isomorphism form an open subset of $X$. All of these assertions are local in nature, and therefore are valid for vector bundles as well as for trivial families. \par \hfill

\textbf{Remark:} The finite dimensionality of $V$ is basic to the previous argument. If one wants to consider infinite dimensional vector bundles, then one must distinguish between the different operator topologies on $\Hom(V, W)$. \newpage

%%%%%%%%%%%%%%%%%%%%%%%%%%%%%%%%%%%%%%%%%%%%%%%%%%%%%%%

\section{Operations on vector bundles.} \label{sec:1.2}  Natural operations on vector spaces, such as direct sum and tensor product, can be extended to vector bundles. The only troublesome question is how one should topologize the resulting spaces. We shall give a general method for extending operations from vector spaces to vector bundles which will handle all of these problems uniformly. \par 

Let $T$ be a functor which carries finite dimensional vector spaces into finite dimensional vector spaces. For simplicity, we assume that $T$ is a covariant functor of one variable. Thus, to every vector space $V$, we have an associated vector space $T(V)$. We shall say that $T$ is a \textit{continuous functor} if for all $V$ and $W$ , the map $T: \Hom(V, W) \to \Hom(T(V), T(W))$ is continuous. \par 

If $E$ is a vector bundle, we define the set $T(E)$ to be the union
\begin{equation*}
    \bigcup_{x \in X} T(E_x) \text{,}
\end{equation*}
and, if $\varphi: E \to F$, we define $T(\varphi): T(E) \to T(F)$ by the maps $T(\varphi_x): T(E_x) \to T(F_x)$. What we must show is that $T(E)$ has a natural topology, and that, in this topology, $T(\varphi_x)$ is continuous. \par 

We begin by defining $T(E)$ in the case that $E$ is a product bundle. If $E = X \times V$, we define $T(E)$ to be $X \times T(V)$ in the product topology. Suppose that $F = X \times W$, and that $\varphi: E \to F$ is a homomorphism. Let $\Phi: X \to \Hom(V, W)$ be the corresponding map. Since, by hypothesis, $T: \Hom(V, W) \to \Hom(T(V), T(W))$ is continuous, $T\Phi: X \to \Hom(T(V), T(W))$ is continuous. Thus $T(\varphi): X \times T(V) \to X \times T(W)$ is also continuous. If $\varphi$ is an isomorphism, then $T\varphi$ will be an isomorphism since it is continuous and an isomorphism on each fiber. \par 

Now suppose that $E$ is trivial, but has no preferred product structure. Choose an isomorphism $\alpha: E \to X \times V$ , and topologize $T(E)$ by requiring $T(\alpha): T(E) \to X \times T(V)$ to be a homeomorphism. If $\beta: E \to X \times W$ is any other isomorphism, by letting $\varphi = \beta \alpha^{-1}$ above, we see that $T(\alpha)$ and $T(\beta)$ induce the same topology on $T(E)$, since $T(\varphi) = T(\beta)T(\alpha)^{-l}$ is a homeomorphism. Thus, the topology on $E$ does not depend on the choice of $\alpha$. Further, if $Y \subset X$, it is clear that the topology on $T(E) \vert Y$ is the same as that on $T(E \vert Y)$. Finally, if $\varphi: E \to F$ is a homomorphism of trivial bundles, we see that $T(\varphi): T(E) \to T(F)$ is continuous, and therefore is a homomorphism. \par

Now suppose that $E$ is any vector bundle. Then if $U \subset X$ is such that $E \vert U$ is trivial, we topologize $T(E \vert U)$ as above. We topologize $T(E)$ by taking for the open sets, those subsets $V \subset T(E)$ such that $V \cap (T(E) \vert U)$ is open in $T(E \vert U)$ for all open $U \subset X$ for which $E \vert U$ is trivial. The reader can now easily verify that if $Y \subset X$, the topology on $T(E \vert Y)$ is the same as that on $T(E) \vert Y$, and that, if $\varphi: E \to F$ is any homomorphism, $T(\varphi): T(E) \to T(F)$ is also a homomorphism. \par 

If $f: Y \to X$ is a continuous map and $E$ is a vector bundle over $X$ then, for any continuous functor $T$, we have a natural isomorphism
\begin{equation*}
    f^* T(E) \cong T f^*(E) \text{.}
\end{equation*}

The case when $T$ has several variables both covariant and contravariant, proceeds similarly. Therefore we can define for vector bundles $E, F$ corresponding bundles:

\begin{enumerate}[(i)]
    \item $E \oplus F$, their direct sum
    \item $E \otimes F$, their tensor product
    \item $\Hom(E, F)$
    \item $E^*$, the dual bundle of E
    \item $\lambda^i(E)$, where $\lambda^i$ is the $i^{\text{th}}$ exterior power.
\end{enumerate}

We also obtain natural isomorphisms

\begin{enumerate}[(i)]
    \item $E \oplus F \cong F \oplus E$
    \item $E \otimes F \cong F \otimes E$
    \item $E \otimes (F' \oplus F'') \cong (E \otimes F') \oplus (E \otimes F'')$
    \item $\Hom(E, F) \cong E^* \otimes F$
    \item $\lambda^k(E \oplus F) \cong \bigoplus_{i+j=k} (\lambda^i(E) \otimes \lambda^j(F))$
\end{enumerate}
\pagebreak

Finally, notice that sections of $\Hom(E, F)$ correspond in a 1 - 1 fashion with homomorphisms $\varphi: E \to F$. We therefore define $\HOM(E, F)$ to be the vector space of all homomorphisms from $E$ to $F$, and make the identification $\HOM(E,F) = \Gamma(\Hom(E,F))$. \newpage

%%%%%%%%%%%%%%%%%%%%%%%%%%%%%%%%%%%%%%%%%%%%%%

\section{Sub-bundles and quotient bundles.}\label{sec:1.3} Let $E$ be a vector bundle. A \textit{sub-bundle} of $E$ is a subset of $E$ which is a bundle in the induced structure. \par 

A homomorphism $\varphi: F \to E$ is called a \textit{monomorphism} (respectively \textit{epimorphism}) if each $\varphi_x: F_x \to E_x$ is a monomorphism (respectively epimorphism). Notice that $\varphi: F \to E$ is a monomorphism if and only if $\varphi^*: F^* \to E^*$ is an epimorphism. If $F$ is a sub-bundle of $E$, and if $\varphi: F \to E$ is the inclusion map, then $\varphi$ is a monomorphism. \par 

\subsection{LEMMA}\label{lem:1.3.1} \textit{ If $\varphi: F \to E$ is a monomorphism, then $\varphi(F)$ is a sub-bundle of $E$, and $\varphi: F \to \varphi(F)$ is an isomorphism.} \par 

\textbf{Proof:} $\varphi: F \to \varphi(F)$ is a bijection, so if $\varphi(F)$ is a sub-bundle, $\varphi$ is an isomorphism. Thus we need only show that $\varphi(F)$ is a sub-bundle. \par 

The problem is local, so it suffices to consider the case when $E$ and $F$ are product bundles. Let $E = X \times V$ and let $x \in X$; choose $W_x \subset V$ to be a subspace complementary to $\varphi(F_x)$. $G = X \times W_x$ is a sub-bundle of E. Define $\theta: F \oplus G \to E$ by $\theta(a \oplus b) = \varphi(a) + i(b)$, where $i: G \to E$ is the inclusion. By construction, $\theta_x$ is an isomorphism. Thus, there exists an open neighborhood $U$ of $x$ such that $\theta \vert U$ is an isomorphism. $F$ is a sub-bundle of $F \oplus G$. so $\theta(F) = \varphi(F)$ is a sub-bundle of $\theta(F \oplus G) = E$ on $U$. \par 

Notice that in our argument, we have shown more than we have stated. We have shown that if $\varphi: F \to E$, then the set of points for which $\varphi_x$ is a monomorphism form an open set. Also, we have shown that, locally, a sub-bundle is direct summand. This second fact allows us to define quotient bundles. \par 

\subsection{DEFINITION}. If $F$ is a sub-bundle of $E$, the quotient bundle $E/F$ is the union of all the vector spaces $E_x/F_x$ given the quotient topology. \par 

Since $F$ is locally a direct summand in $E$, we see that $E/F$ is locally trivial, and thus is a bundle. This justifies the terminology. \par 

If $\varphi: F \to E$ is an arbitrary homomorphism, the function $\dimension(\kernel(\varphi_x))$ need not be constant, or even locally constant. \par 

\subsection{DEFINITION} $\varphi: F \to E$ is said to be a \textit{strict} homomorphism if $\dimension(\kernel(\varphi_x))$ is locally constant. \par 

\subsection{PROPOSITION}. If $\varphi: F \to E$ is strict, then:
\begin{enumerate}[(i)]
    \item $\kernel(\varphi) = \bigcup_x \kernel(\varphi_x)$ is a sub-bundle of F
    \item $\image(\varphi) = \bigcup_x \image(\varphi_x)$ is a sub-bundle of E
    \item $\cokernel(\varphi) = \bigcup_x \cokernel(\varphi_x)$ is a bundle in the quotient structure.
\end{enumerate}

\textbf{Proof:} Notice that (ii) implies (iii). We first prove (ii). The problem is local, so we can assume $F = X \times V$ for some $V$. Given $x \in X$ , we choose $W_x \subset V$ complementary to $\ker(\varphi_x)$ in $V$. Put $G = X \times W_x$; then $\varphi$ induces, by composition with the inclusion, a homomorphism, $\psi: G \to E$, such that $\psi_x$ is a monomorphism. Thus, $\psi$ is a monomorphism in some neighborhood $U$ of $x$. Therefore, $\psi(G) \vert U$ is a sub-bundle of $E \vert U$. However, $\psi(G) \subset \varphi(F)$, and since $\dim(\varphi(F_y))$ is constant for all $y$, and $\dim(\psi(G_y)) = \dim(\psi(G_x)) = \dim(\varphi(F_x)) = \dim(\varphi(F_y))$ for all $y \in U$, $\psi(G) \vert U = \varphi(F) \vert U$. Thus $\varphi(F)$ is a sub-bundle of E. \par 

Finally, we must prove (i). Clearly, $\varphi^*: E^* \to F^*$ is strict. Since $F^* \to \coker(\varphi^*)$ is an epimorphism, $ (\coker(\varphi^*))^* \to F^{**} $ is a monomorphism. However, for each $x$ we have a natural commutative diagram:

\begin{center}
\begin{tikzcd}
\ker(\varphi_x) \arrow[d] \arrow[r] & F_x \arrow[d] \\
(\coker \varphi_x^*)^* \arrow[r]    & F_x^{**}     
\end{tikzcd}
\end{center}

in which the vertical arrows are isomorphisms. Thus $\ker(\varphi) \cong (\coker(\varphi^*))^*$ and so, by (\ref{lem:1.3.1}), is a sub-bundle of $F$. \par 

Again, we have proved something more than we have stated. Our argument shows that for any $x \in X$, $\dim \varphi_x(F_x) \leq \dim \varphi_y(F_y)$ for all $y \in U$, $U$ some neighborhood of $x$. Thus, \textit{$\rank(\varphi_x)$ is an upper semi-continuous function of $x$}. \par 

\subsection{DEFINITION} A projection operator $P: E \to E$ is a homomorphism such that $P^2 = P$. \par

Notice that $\rank(P_x) + \rank(1 - P_x) = \dim E_x$ so that, since both $\rank(P_x)$ and $\rank(1 - P_x)$ are upper semi-continuous functions of $x$, they are locally constant. Thus both $P$ and $1 - P$ are strict homomorphisms. Since $\ker(P) = (1 - P)E$, $E$ is the direct sum of the two sub-bundles $PE$ and $(1 - P)E$. Thus any projection operator $P: E \to E$ determines a direct sum decomposition $E = (PE) \oplus ((1 - P)E)$. \par 

We now consider metrics on vector bundles. We define a functor $\Herm$ which assigns to each vector space $V$ the vector space $\Herm(V)$ of all Hermitian forms on $V$. By the techniques of \cref{sec:1.2}, this allows us to define a vector bundle $\Herm(E)$ for every bundle $E$. \par 

\subsection{DEFINITION} A \textit{metric} on a bundle $E$ is any section $h: X \to \Herm(E)$ such that $h(x)$ is positive definite for all $x \in X$. A bundle with a specified metric is called a Hermitian bundle. \par 

Suppose that $E$ is a bundle, $F$ is a sub-bundle of $E$, and that $h$ is a Hermitian metric on $E$. Then for each $x \in X$ we consider the orthogonal projection $P_x: E_x \to F_x$ defined by the metric. This defines a map $P: E \to F$ which we shall now check is continuous. The problem being local we may assume $F$ is trivial, so that we have sections $f_1, \ldots, f_n$ of $F$ giving a basis in each fiber. Then for $v \in F_x$ we have
\begin{equation*}
    P_x(v) = \sum_i h_x(v, f_i(x))f_i(x)
\end{equation*}

Since $h$ is continuous this implies that $P$ is continuous. Thus P is a projection operator on $E$. If $F_x^\bot$ is the subspace of $E_x$ which is orthogonal to $F$ under $h$, we see that $F^\bot = \bigcup_x F_x^\bot$ is the kernel of $P$, and thus is a sub-bundle of $E$, and that $E \cong F \oplus F^\bot$. Thus, a metric provides any sub-bundle with a definite complementary sub-bundle. \par \pagebreak

\textbf{Remark:} So far, most of our arguments have been of a very general nature, and we could have replaced "continuous" with "algebraic", "differentiable", "analytic", etc. without any trouble. In the next section, our arguments become less general. \newpage

%%%%%%%%%%%%%%%%%%%%%%%%%%%%%%%%%%%%%%%%%%%%%%

\section{Vector bundles on compact spaces. }In order to proceed further, we must make some restriction on the sort of base spaces which we consider. We shall assume from now on that our base spaces are \textit{compact Hausdorff}. We leave it to the reader to notice which results hold for more general base spaces. \par 

Recall that if $f: X \to V$ is a continuous vector-valued function the support of $f$ (written $\supp f$) is the closure of $f^{-1}(V - \{0\})$. \par

We need the following results from point set topology. We state them in vector forms which are clearly equivalent to the usual forms: \par \hfill

\textbf{Tietze Extension Theorem.} Let $X$ be a normal space, $Y \subset X$ a closed subspace, $V$ a real vector space, and $f: Y \to V$ a continuous map. Then there exists a continuous map $g: X \to V$ such that $g \vert Y = f$. \par 

\textbf{Existence of Partitions of Unity.} Let $X$ be a compact Hausdorff space, $\{U_i\}$ a finite open covering. Then there exist continuous maps $f_i : X \to \mathbb{R}$ such that:
\begin{enumerate}
    \item $f_i \geq 0$ \qquad all $x \in X$
    \item $\supp (f_i) \subset U_i$
    \item $\sum_i f_i(x) = 1$ \qquad all $x \in X$
\end{enumerate}

Such a collection $\{f_i\}$ is called a \textit{partition of unity}. \par
We first give a bundle form of the Tietze extension theorem. \par

\subsection{LEMMA}\label{lem:1.4.1} \textit{Let $X$ be compact Hausdorff, $Y \subset X$ a closed subspace, and $E$ a bundle over $X$. Then any section $s: Y \to E \vert Y$ can be extended to $X$.} \par 

\textbf{Proof:} Let $s \in \Gamma(E \vert Y)$. Since, locally, $s$ is a vector-valued function, we can apply the Tietze extension theorem to show that for each $x \in X$, there exists an open set $U$ containing $x$ and $t \in \Gamma(E \vert U)$ such that $t \vert U \cap Y = s \vert U \cap Y$. Since $X$ is compact, we can find a finite subcover $\{U_\alpha \}$ by such open sets. Let $t_\alpha \in \Gamma(E \vert U_\alpha)$ be the corresponding sections and let $\{p_\alpha\}$ be a partition of unity with $\supp(p_\alpha) \in U_\alpha$. We define $S_\alpha \in \Gamma(E)$ by
\begin{equation*}
S_\alpha =
    \begin{cases}
        p_\alpha(x) t_\alpha(x) & \text{if $x \in U_\alpha$} \\
        0 & \text{otherwise.}
    \end{cases}
\end{equation*}
Then $\sum S_\alpha$ is a section of $E$ and its restriction to $Y$ is clearly $s$.

\subsection{LEMMA}\label{lem:1.4.2} \textit{Let $Y$ be a closed subspace of a compact Hausdorff space $X$, and let $E, F$ be two vector bundles over $X$. If $f: E \vert Y \to F \vert Y$ is an isomorphism, then there exists an opet set $U$ containing $Y$ and an extension $f: E \vert U \to F \vert U$ which is an isomorphism.} \par 

\textbf{Proof:} $f$ is a section of $\Hom(E \vert Y, F \vert Y)$, and thus, extends to a section of $\Hom(E, F)$. Let $U$ be the set of those points for which this map is an isomorphism. Then $U$ is open and contains $Y$.

\subsection{LEMMA}\label{lem:1.4.3} \textit{Let $Y$ be a compact Hausdorff space, $f_t: Y \to X \\ (0 \leq t \leq 1)$ a homotopy and $E$ a vector bundle over $X$. Then}
\begin{equation*}
    f_0^*E \cong f_1^*E.
\end{equation*}

\textbf{Proof:} If $I$ denotes the unit interval, let $f: Y \times I \to X$ be the homotopy, so that $f(y, t) = f_t(y)$, and let $\pi: Y \times I \to Y$ denote the projection. Now apply (\ref{lem:1.4.2}) to the bundles $f^*E, \pi^* f_t^* E$ and the subspace $Y \times \{t\}$ of $Y \times I$, on which the re is an obvious isomorphism $s$. By the compactness of $Y$ we deduce that $f^*E$ and $\pi^* f_t^* E$ are isomorphic in some strip $Y \times \delta t$ where $\delta t$ denotes a neighborhood of $\{t\}$ in $I$. Hence the isomorphism class of $f_t^*E$ is a locally constant function of $t$. Since $I$ is connected this implies it is constant, whence
\begin{equation*}
f_0^*E \cong f_1^*E.
\end{equation*}
We shall use $\Vect(X)$ to denote the set of isomorphism classes of vector bundles on $X$, and $\Vect_n(X)$ to denote the subset of $\Vect(X)$ given by bundles of dimension $n$. $\Vect(X)$ is an abelian semi-group under the operation $\oplus$. In $\Vect_n(X)$ we have one naturally distinguished element - the class of the trivial bundle of dimension $n$.

\subsection{LEMMA}\label{lem:1.4.4}
\begin{enumerate}[(1)]
    \item \textit{If $f: X \to Y$ is a homotopy equivalence, $f^*: \Vect(Y) \to \Vect(X)$ is bijective.}
    \item \textit{If $X$ is contractible, every bundle over $X$ is trivial and $\Vect(X)$ is isomorphic to the non-negative integers.}
\end{enumerate}

\subsection{LEMMA}\label{lem:1.4.5} \textit{If $E$ is a bundle over $X \times I$, and $\pi: X \times I \to X \times \{0\}$ is the projection, $E$ is isomorphic to $\pi^*(E \vert X \times \{0\})$.} \par 

Both of these lemmas are immediate consequences of (\ref{lem:1.4.3}). \par \hfill

Suppose now $Y$ is closed in $X$, $E$ is a vector bundle over $X$ and $\alpha: E \vert Y \to Y \times V$ is an isomorphism. We refer to $\alpha$ as a \textit{trivialization of $E$ over $Y$}. Let $\pi: Y \times V \to V$ denote the projection and define an equivalence relation on $E \vert Y$ by
\begin{equation*}
    e \sim e' \iff \pi \alpha (e) = \pi \alpha (e')
\end{equation*}

We extend this by the identity on $E \vert X - Y$ and we let $E/\alpha$ denote the quotient space of $E$ given by this equivalence relation. It has a natural structure of a family of vector spaces over $X/Y$. We assert that $E/\alpha$ is in fact a vector bundle. To see this we have only to verify the local triviality at the base point $Y/Y$ of $X/Y$. Now by (\ref{lem:1.4.2}) we can extend $\alpha$ to an isomorphism $\alpha: E \vert U \to U \times V$ for some open set $U$ containing $Y$. Then $\alpha$ induces an isomorphism
\begin{equation*}
    (E \vert U) / \alpha \cong (U/Y) \times V
\end{equation*}
which establishes the local triviality of $E/\alpha$. \par

Suppose $\alpha_0, \alpha_1$ are homotopic trivializations of $E$ over $Y$. This means that we have a trivialization $\beta$ of $E \times I$ over $Y \times I \subset X \times I$ inducing $\alpha_0$ and $\alpha_1$ at the two end points of $I$. Let $f: (X \times Y) \times I \to (X \times I)/(Y \times I)$ be the natural map. Then $f^*(E \times I/\beta)$ is a bundle on $(X/Y) \times I$ whose restriction to $(X/Y) \times \{i\}$ is $E/\alpha_i \quad (i = 0, 1)$. Hence, by (\ref{lem:1.4.3}),
\begin{equation*}
    E/\alpha_0 \cong  E/\alpha_1.
\end{equation*}

To summarize we have established \par

\subsection{LEMMA}\label{lem:1.4.7} \textit{A trivialization $\alpha$ of a bundle $E$ over $Y \subset X$ defines a bundle $E/\alpha$ over $X/Y$. The isomorphism class of $E/\alpha$ depends only on the homotopy class of $\alpha$.} \par 

Using this we shall now prove \par

\subsection{LEMMA}\label{lem:1.4.8} \textit{Let $Y \subset X$ be a closed contractible subspace. Then $f: X \to X/Y$ induces a bijection $f^*: \Vect(X) \to \Vect(X/Y)$.} \par 

\textbf{Proof:} Let $E$ be a bundle on $X$ then by (\ref{lem:1.4.4}) $E \vert Y$ is trivial. Thus trivializations $\alpha: E \vert Y \to Y \times V$ exist. Moreover, two such trivializations differ by an automorphism of $Y \times V$, i.e., by a map $Y \to \GL(V)$. But $\GL(V) = \GL(n, \mathbb{C})$ is connected and $V$ is contractible. Thus $\alpha$ is unique up to homotopy and so the isomorphism class of $E \vert \alpha$ is uniquely determined by that of $E$. Thus we have constructed a map
\begin{equation*}
    \Vect(X) \to \Vect(X/Y)
\end{equation*}
and this is clearly a two-sided inverse for $f^*$. Hence $f^*$ is bijective as asserted. \par \hfill

Vector bundles are frequently constructed by a glueing or clutching construction which we shall now describe. Let
\begin{equation*}
    X = X_1 \cup X_2, \qquad A = X_1 \cap X_2,
\end{equation*}
all the spaces being compact. Assume that $E_i$ is a vector bundle over $X_i$ and that $\varphi: E_1 \vert A \to E_2 \vert A $ is an isomorphism. Then we define the vector bundle $E_1 \cup_\varphi E_2$ on $X$ as follows. As a topological space $E_1 \cup_\varphi E_2$ is the quotient of the disjoint sum $E_1 + E_2$ by the equivalence relation which identifies $e_1 \in E_1 \vert A$ with $\varphi(e_1) \in E_2 \vert A$. Identifying $X$ with the corresponding quotient of $X_1 + X_2$ we obtain a natural projection $p: E_1 \cup_\varphi E_2 \to X$, and $p^{-1}(x)$ has a natural vector space structure. It remains to show that $E_1 \cup_\varphi E_2$ is locally trivial. Since $E_1 \cup_\varphi E_2 \vert X - A = (E_1 \vert X_1 - A) + (E_2 \vert X_2 - A)$ the local triviality at points $x \notin A$ follows from that of $E_1$ and $E_2$. Therefore, let $a \in A$ and let $V_1$ be a closed neighborhood of $a$ in $X_1$ over which $E_1$ is trivial. so that we have an isomorphism
\begin{equation*}
    \theta_1: E_1 \vert V_1 \to V_1 \times \mathbb{C}^n.
\end{equation*}
Restricting to $A$ we get an isomorphism
\begin{equation*}
    \theta_1^A: E_1 \vert V_1 \cap A \to (V_1 \cap A) \times \mathbb{C}^n.
\end{equation*}
Let
\begin{equation*}
    \theta_2^A: E_2 \vert V_1 \cap A \to (V_1 \cap A) \times \mathbb{C}^n.
\end{equation*}
be the isomorphism corresponding to $\theta_1^A$ under $\varphi$. By (\ref{lem:1.4.2}) this can be extended to an isomorphism
\begin{equation*}
    \theta_2: E_2 \vert V_2 \to V_2 \times \mathbb{C}^n.
\end{equation*}
where $V_2$ is a neighborhood of $a$ in $X_2$. The pair $\theta_1, \theta_2$ then defines in an obvious way an isomorphism
\begin{equation*}
    \theta_1 \cup_\varphi \theta_2: E_1 \cup_\varphi E_2 \vert V_1 \cup V_2 \to (V_1 \cup V_2) \times \mathbb{C}^n.
\end{equation*}
establishing the local triviality of $E_1 \cup_\varphi E_2$. \par 

Elementary properties of this construction are the following:
\begin{enumerate}[(i)]
    \item If $E$ is a bundle over $X$ and $E_i = E \vert X_i$ then the identity defines an isomorphism $I_A: E_1 \vert A \to E_2 \vert A$. and
    \begin{equation*}
        E_1 \cup_{I_A} E_2 \cong E.
    \end{equation*}
    \item If $\beta_i: E_i \to E_i'$ are isomorphisms on $X_i$ and $\varphi'\beta_1 = \beta_2\varphi$, then
    \begin{equation*}
        E_1 \cup_\varphi E_2 \cong  E_1' \cup_{\varphi'} E_2'.
    \end{equation*}
    \item $(E_i, \varphi)$ and $(E_i', \varphi')$ are two "clutching data" on the $X_i$, then
    \begin{equation*}
        (E_1 \cup_\varphi E_2) \oplus (E_1' \cup_{\varphi'} E_2') \cong E_1 \oplus E_1' \cup_{\varphi \oplus \varphi'} E_2 \oplus E_2',
    \end{equation*}
    \begin{equation*}
        (E_1 \cup_\varphi E_2) \otimes (E_1' \cup_{\varphi'} E_2') \cong E_1 \otimes E_1' \cup_{\varphi \otimes \varphi'} E_2 \otimes E_2',
    \end{equation*}
    \begin{equation*}
        (E_1 \cup_\varphi E_2)^* \cong E_1^* \cup_{(\varphi^*)^{-1}} E_2^*
    \end{equation*}
\end{enumerate}

Moreover, we also have

\subsection{LEMMA}\label{lem:1.4.6} \textit{The isomorphism class of $E_1 \cup_\varphi E_2$ depends only on the homotopy class of the isomorphism $\varphi: E_1 \vert A \to E_2 \vert A$.} \par 

\textbf{Proof:} A homotopy of isomorphism $E_1 \vert A \to E_2 \vert A$ means an isomorphism
\begin{equation*}
    \Phi: \pi^* E_1 \vert A \times I \to \pi^* E_2 \vert A \times I
\end{equation*}
where I is the unit interval and $\pi: X \times I \to X$ is the projection. \par 

Let 
\begin{equation*}
    f_t: X \to X \times I
\end{equation*}
be defined by $f_t(x) = x \times \{t\}$ and denote by
\begin{equation*}
    \varphi_t: E_1 \vert A \to E_2 \vert A
\end{equation*}
the isomorphism induced from $\Phi$ by $f_t$. Then
\begin{equation*}
    E_1 \cup_{\varphi_t} E_2 \cong f_t^*(\pi^* E_1 \cup_\Phi \pi^* E_2)
\end{equation*}
Since $f_0$ and $f_1$ are homotopic it follows from (\ref{lem:1.4.3}) that
\begin{equation*}
    E_1 \cup_{\varphi_0} E_2 \cong E_1 \cup_{\varphi_1} E_2
\end{equation*}
as required. \par 

\textbf{Remark:} The "collapsing" and "clutching" constructions for bundles (on $X/Y$ and $X_1 \cup X_2$ respectively) are both special cases of a general process of forming bundles over quotient spaces We leave it as an exercise to the reader to give a precise general formulation. \par 

We shall denote by $[X, Y]$ the set of homotopy classes of maps $X \to Y$. \par 

\subsection{LEMMA}\label{lem:1.4.9} \textit{For any $X$, there is a natural isomorphism $\Vect_n(S(X)) \cong [X, \GL(n, \mathbb{C})]$.} \par 

\textbf{Proof:} Write $S(X)$ as $C^+(X) \cup C^-(X)$, where $C^+(X) = [0, \frac{1}{2}] \times X/\{0\} \times X$, $C^-(X) = [\frac{1}{2}, 1] \times X/\{1\} \times X$. Then $C^+(X) \cap C^-(X) = X$. If $E$ is any $n$-dimensional bundle over $S(X)$, $E \vert C^+(x)$ and $E \vert C^-(x)$ are trivial. Let $\alpha^\pm: E \vert C^\pm(X) \cong C^\pm(X) \times V$ be such isomorphisms. Then $(\alpha^+ \vert X)(\alpha^- \vert X)^{-1}: X \times V \to X \times V$ is a bundle map, and thus defines a map $\alpha$ of $X$ into $\GL(n, \mathbb{C}) = \Iso(V)$. Since both $C^+(X)$ and $C^-(X)$ are contractible, the homotopy classes of both $\alpha^+$ and $\alpha^-$ are well defined, and thus the homotopy class of $\alpha$ is well defined. Thus we have a natural map $\theta: \Vect_n(S(X)) \to [X, \GL(n, \mathbb{C})]$. The clutching construction on the other hand defines by (\ref{lem:1.4.6}) a map
\begin{equation*}
    \varphi: [X, \GL(n, \mathbb{C})] \to \Vect_n(S(X))
\end{equation*}
It is clear that $\theta$ and $\varphi$ are inverses of each other and so are bijections. \par 

We have just seen that $\Vect_n(S(X))$ has a homotopy theoretic interpretation. We now give a similar interpretation to $\Vect_n(X)$. First we must establish some simple facts about quotient bundles. \par 

\subsection{LEMMA}\label{lem:1.4.10} \textit{Let $E$ be any bundle over $X$. Then there exists a (Hermitian) metric on $E$.} \par 

\textbf{Proof:} A metric on a vector space $V$ defines a metric on the product bundle $X \times V$. Hence metrics exist on trivial bundles. Let $\{U_\alpha\}$ be a finite open covering of $X$ such that $E \vert U_\alpha$ is trivial and let $h_\alpha$ be a metric for $E \vert U_\alpha$. Let $\{p_\alpha\}$ be a partition of unity with $\supp p_\alpha \subset U_\alpha$ and define
\begin{equation*}
k_\alpha(x) =
    \begin{cases}
        p_\alpha(x) h_\alpha(x) & \text{for $x \in U_\alpha$} \\
        0 & \text{otherwise.}
    \end{cases}
\end{equation*}

Then $k_\alpha$ is a section of $\Herm(E)$ and is positive semi-definite. But for any $x \in X$ there exists $\alpha$ such that $p_\alpha(x) > 0$ (since $\sum p_\alpha = 1$) and so $x \in U_\alpha$. Hence, for this $\alpha$, $k_\alpha(x)$ is positive definite. Hence $\sum_\alpha k_\alpha(x)$ is positive definite for all $x \in X$ and so $\sum k_\alpha$ is a metric for $E$. \par 

A sequence of vector bundle homomorphisms
\begin{equation*}
    \begin{tikzcd}
    {} \arrow[r] & E \arrow[r] & F \arrow[r] & \ldots
    \end{tikzcd}
\end{equation*}
is called \textit{exact} if for each $x \in X$ the sequence of vector space homomorphisms
\begin{equation*}
    \begin{tikzcd}
    {} \arrow[r] & E_x \arrow[r] & F_x \arrow[r] & \ldots
    \end{tikzcd}
\end{equation*}
is exact. \par 

\subsection{COROLLARY} Suppose that \begin{tikzcd} 0 \arrow[r] & E' \arrow[r, "\varphi'"] & E \arrow[r, "\varphi''"] & E'' \arrow[r] & 0\end{tikzcd} is an exact sequence of bundles over $X$. Then there exists an homorphism $E \cong E' \oplus E"$. \par 

\textbf{Proof:} Give $E$ a metric. Then $E \cong E' \oplus (E')^\bot$. However, $(E')^\bot \cong E''$. \par \hfill

A subspace $V \subset \Gamma(E)$ is said to be \textit{ample} if
\begin{equation*}
    \varphi: X \times V \to E
\end{equation*}
is a surjection, where $\varphi(x, s) = s(x)$.

\subsection{LEMMA}\label{1.4.12} \textit{If $E$ is any bundle over a compact Hausdorff space $X$, then $\Gamma(E)$ contains a finite dimensional ample subspace.} \par 

\textbf{Proof:} Let $\{U_\alpha\}$ be a finite open covering of $X$ so that $E \vert U_\alpha$ is trivial for each $\alpha$, and let $\{p_\alpha\}$ be a partition of unity with $\supp p_\alpha \subset U_\alpha$. Since $E \vert U_\alpha$ is trivial we can find a finite-dimensional ample subspace $V_\alpha \subset \Gamma(E \vert U_\alpha)$. Now define
\begin{equation*}
    \theta_\alpha: V_\alpha \to \Gamma(E)
\end{equation*}
by
\begin{equation*}
\theta_\alpha v_\alpha(x) =
    \begin{cases}
        p_\alpha(x) v_\alpha(x) & \text{if $x \in U_\alpha$} \\
        0 & \text{otherwise.}
    \end{cases}
\end{equation*}

The $\theta_\alpha$ define a homomorphism
\begin{equation*}
    \theta: \prod_\alpha V_\alpha \to \Gamma(E)
\end{equation*}
and the image of $\theta$ is a finite dimensional 8ubspace of $\Gamma(E)$; in fact, for each $x \in X$ there exists $\alpha$ with $p_\alpha(x) > 0$ and so the map
\begin{equation*}
    \theta_\alpha(V_\alpha) \to E_x
\end{equation*}
is surjective. \par 

\subsection{COROLLARY} \textit{If $E$ is any bundle, there exists an epimorphism $\varphi: X \times \mathbb{C}^m \to E$ for some integer $m$.}

\subsection{COROLLARY} \textit{If $E$ is any bundle, there exists a bundle $F$ such that $E \oplus F$ is trivial.} \par 

We are now in a position to prove the existence of a homotopy theoretic definition for $\Vect_n(X)$. We first introduce Grassmann manifolds. If $V$ is any vector space, and $n$ any integer, the set $G_n(V)$ is the set of all subspaces of $V$ of codimension $n$. If $V$ is given some Hermitian metric, each subspace of $V$ determines a projection operator. This defines a map $G_n(V) \to \End(V)$, where $\End(V)$ is the set of endomorphisms of $V$. We give $G_n(V)$ the topology induced by this map. \par 

Suppose that $E$ is a bundle over a space $X$, $V$ is a vector space, and $\varphi: X \times V \to E$ is an epimorphism. If we map $X$ into $G_n(V)$ by assigining to $x$ the subspace $\ker(\varphi_x)$ this map is continuous for any metric on $V$ (here $n = \dim(E)$). We call the map $X \to G_n(V)$ the map induced by $\varphi$. \par 

Let $V$ be a vector space, and let $F \subset G_n(V) \times V$ be the sub-bundle consisting of all points $(g, v)$ such that $v \in g$. Then, if $E = (G_n(V) \times V)/F$ is the quotient bundle, $E$ is called the \textit{classifying bundle} over $G_n(V)$. \par 

Notice that if $E'$ is a bundle over $X$, and $\varphi: X \times V \to E'$ is an epimorphism, then if $f: X \to G_n(V)$ is the map induced by $\varphi$, we have $E' \cong f^*(E)$, where $E$ is the classifying bundle. \par 

Suppose that $h$ is a metric on $V$. We denote by $G_n(V_h)$ the set $G_n(V)$ with the topology induced by $h$. If $h'$ is another metric on $V$, then the epimorphism  $G_n(V) \times V \to E$ (where $E$ is the classifying bundle) induces the identity map  $G_n(V_h) \to G_n(V_{h'})$. Thus the identity map is continuous. Thus, the topology on $G_n(V)$ does not depend on the metric. \par 

Now consider the natural projections
\begin{equation*}
    \mathbb{C}^m \to \mathbb{C}^{m-1}
\end{equation*}
given by $(z_1, \ldots, z_m) \to (z_1, \ldots, z_{m-1})$. These induce continuous maps
\begin{equation*}
    \iota_{m-1}: G_n(\mathbb{C}^m) \to G_n(\mathbb{C}^{m-1}).
\end{equation*}

If $E_{(m)}$ denotes the classifying bundle over $G_n(\mathbb{C}^m)$ it is
immediate that
\begin{equation*}
    \iota_{m-1}^*(E_{(m)}) \cong E_{(m-1)}
\end{equation*}

\subsection{THEOREM}\label{the:1.4.15} \textit{The map}
\begin{equation*}
    \lim_{\to m} [X, G_n(\mathbb{C}^m)] \to \Vect_n(X)
\end{equation*}
\textit{induced by $f \to f^*(E_{(m)})$ for $f: X \to G_n(\mathbb{C}^m)$, is an isomorphism for all compact Hausdorff spaces $X$.}
