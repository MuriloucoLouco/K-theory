\chapter{\scshape $K$-Theory}

\section{Definitions.}\label{sec:2.1} If $X$ is any space, the set $\Vect(X)$ has the structure of an abelian semigroup, where the additive structure is defined by direct sum. If $A$ is any abelian semigroup, we can associate to $A$ an abelian group $K(A)$ with the following property: there is a semigroup homomorphism $\alpha: A \to K(A)$ such that if $G$ is any group, $\gamma: A \to G$ any semigroup homomorphism there is a unique homomorphism $\chi: K(A) \to G$ such that $\gamma = \chi \alpha$. If such a $K(A)$ exists, it must be unique. \par

The group $K(A)$ is defined in the usual fashion. Let $F(A)$ be the free abelian group generated by the elements of $A$, let $E(A)$ be the subgroup of $F(A)$ generated by those elements of the form $a+a' - (a \oplus a')$ where $\oplus$ is the addition in $A$, $a, a' \in A$. Then $K(A) = F(A)/E(A)$ has the universal property described above, with $\alpha: A \to K(A)$ being the obvious map. \par

A slightly different construction of $K(A)$ which is sometimes convenient is the following. Let $\Delta: A \to A \times A$ be the diagonal homomorphism of semi-groups, and let $K(A)$ denote the set of cosets of $\Delta(a)$ in  $A \times A$. It is, a quotient semi-group, but the interchange of factors in $A \times A$ induces an inverse in $K(A)$ so that $K(A)$ is a group. We then define $\alpha_A: A \to K(A)$ to be the composition of $a \to (a, 0)$ with the natural projection $A \times A \to K(A)$ (we assume $A$ has a zero for simplicity). The pair $(K(A), \alpha_A)$ is a functor of $A$ so that if $\gamma: A \to B$ is a semi-group homomorphism we have a commutative diagram
\begin{center}
\begin{tikzcd}
A \arrow[rr, "\alpha_A"] \arrow[dd, "\gamma"'] &  & K(A) \arrow[dd, "K(\gamma)"] \\
                                               &  &                              \\
B \arrow[rr, "\alpha_B"]                       &  & K(B)                        
\end{tikzcd}
\end{center}
If $B$ is a group $\alpha_B$ is an isomorphism. That shows $K(A)$ has the required universal property. \par 

If $A$ is also a semi-ring (that is, $A$ possesses multiplication which is distributive over the addition of $A$) then $K(A)$ is clearly a ring. \par \hfill

