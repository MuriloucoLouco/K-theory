\chapter{\scshape $K$-Theory}

\section{Definitions.}\label{sec:2.1} If $X$ is any space, the set $\Vect(X)$ has the structure of an abelian semigroup, where the additive structure is defined by direct sum. If $A$ is any abelian semigroup, we can associate to $A$ an abelian group $K(A)$ with the following property: there is a semigroup homomorphism $\alpha: A \to K(A)$ such that if $G$ is any group, $\gamma: A \to G$ any semigroup homomorphism there is a unique homomorphism $\chi: K(A) \to G$ such that $\gamma = \chi \alpha$. If such a $K(A)$ exists, it must be unique. \par

The group $K(A)$ is defined in the usual fashion. Let $F(A)$ be the free abelian group generated by the elements of $A$, let $E(A)$ be the subgroup of $F(A)$ generated by those elements of the form $a+a' - (a \oplus a')$ where $\oplus$ is the addition in $A$, $a, a' \in A$. Then $K(A) = F(A)/E(A)$ has the universal property described above, with $\alpha: A \to K(A)$ being the obvious map. \par

A slightly different construction of $K(A)$ which is sometimes convenient is the following. Let $\Delta: A \to A \times A$ be the diagonal homomorphism of semi-groups, and let $K(A)$ denote the set of cosets of $\Delta(a)$ in  $A \times A$. It is, a quotient semi-group, but the interchange of factors in $A \times A$ induces an inverse in $K(A)$ so that $K(A)$ is a group. We then define $\alpha_A: A \to K(A)$ to be the composition of $a \to (a, 0)$ with the natural projection $A \times A \to K(A)$ (we assume $A$ has a zero for simplicity). The pair $(K(A), \alpha_A)$ is a functor of $A$ so that if $\gamma: A \to B$ is a semi-group homomorphism we have a commutative diagram
\begin{center}
\begin{tikzcd}
A \arrow[rr, "\alpha_A"] \arrow[dd, "\gamma"'] &  & K(A) \arrow[dd, "K(\gamma)"] \\
                                               &  &                              \\
B \arrow[rr, "\alpha_B"]                       &  & K(B)                        
\end{tikzcd}
\end{center}
If $B$ is a group $\alpha_B$ is an isomorphism. That shows $K(A)$ has the required universal property. \par 

If $A$ is also a semi-ring (that is, $A$ possesses multiplication which is distributive over the addition of $A$) then $K(A)$ is clearly a ring. \par \hfill

If $X$ is a space, we write $K(X)$ for the ring $K(\Vect(X))$. No confusion should result from this notation. If $E \in \Vect(X)$, we shall write $[E]$ for the image of $E$ in $K(X)$. Eventually, to avoid excessive notation, we may simply write $E$ instead of $[E]$ when there is no danger of confusion. \par 

Using our second construction of $K$ it follows that, if $X$ is a space, every element of $K(X)$ is of the form $[E] - [F]$, where $E, F$ are bundles over $X$. Let $G$ be a bundle such that $F \oplus G$ is trivial. We write $\underline{n}$ for the trivial bundle of dimension $n$ . Let $F \oplus G = \underline{n}$. Then $[E] - [F] = [E] + [G] - ([F] + [G]) = [E \oplus G] - [\underline{n}]$. Thus, every element of $K(X)$ is of the form $[H] -  [\underline{n}]$. \par 

Suppose that $E, F$ are such that $[E] = [F]$, then again from our second construction of $K$ it follows that there is a bundle $G$ such that $E \oplus G \cong F \oplus G$. Let $G'$ be a bundle such that $G \oplus G' \cong \underline{n}$. Then $E \oplus G \oplus G' \cong F \oplus G \oplus G'$, so $E \oplus \underline{n} \cong F \oplus \underline{n}$. If two bundles become equivalent when a suitable trivial bundle is added to each of them, the bundles are said to be \textit{stably equivalent}. Thus, $[E] = [F]$ if and only if $E$ and $F$ are stably equivalent. \par 

Suppose $f: X \to Y$ is a continuous map. Then $f^*: \Vect(Y) \to \Vect(X)$ induces a ring homomorphism $f^*: K(Y) \to K(X)$. By (\ref{lem:1.4.3}) this homomorphism depends only on the homotopy class of $f$. \newpage

%%%%%%%%%%%%%%%%%%%%%%%%%%%%%%%%%%%%%%%%%%%%

\section{The periodicity theorem.}\label{sec:2.2} The fundamental theorem for $K$-theory is the periodicity theorem. In its simplest form, it states that for any $X$, there is an isomorphism between $K(X) \otimes K(S^2)$ and $K(X \times S^2)$. This is a special case of a more general theorem which we shall prove. \par 

If $E$ is a vector bundle over a space $X$, and if $E_0 = E - X$ where $X$ is considered to lie in $E$ as the zero section, the non-zero complex numbers act on $E_0$ as a group of fiber preserving automorphisms. Let $P(E)$ be the quotient space obtained from $E_0$ by dividing by the action of the complex number. $P(E)$ is called the projective bundle associated to $E$. If $p: P(E) \to X$ is the projection map, $p^{-1}(x)$ is a complex projective space for all $x \in X$. If $V$ is a vector space, and $W$ is a vector space of dimension one, $V$ and $V \otimes W$ are isomorphic, but not naturally isomorphic. For any non-zero element $\omega \in W$ the map $\nu \to \nu \otimes \omega$ defines an isomorphism between $V$ and $V \otimes W$, and thus defines an isomorphism $P(\omega): P(V) \to P(V \otimes W)$. However, if $\omega'$ is any other non-zero element of $W$, $\omega' = \lambda \omega$ for some non-zero complex number $\lambda$. Thus $P(\omega) = P(\omega')$, so the isomorphism between $P(V)$ and $P(V \otimes W)$ is natural. Thus, if $E$ is any vector bundle, and $L$ is a line bundle, there is a natural isomorphism $P(E) \cong P(E \otimes L)$. \par 

If $E$ is a vector bundle over $X$ then each point $a \in P(E)_x = P(E_x)$ represents a one-dimensional subspace $H^*_x \subset E_x$. The union of all these defines a subspace $H^* \subset p^*E$, where $p: P(E) \to X$ is the projection. It is easy to check that $H^*$ is a sub-bundle of $p^*E$. In fact, the problem being local we may assume $E$ is a product and then we are reduced to a special case of the Grassmannian already discussed in \cref{sec:1.4}. We have denoted our line-bundle by $H^*$ because we want its dual $H$ (the choice of convention here is dictated by algebro-geometric considerations which we do not discuss here). \par 

We can now state the periodicity theorem. \par 

\subsection{THEOREM}\label{the:2.2.1} \textit{Let $L$ be a line bundle over $X$. Then, as a $K(X)$-algebra $K(P(L \oplus 1))$ is generated by $[H]$, and is subject to the single relation \linebreak $([H] - [1])([L][H] - [1]) = 0$.} \par 

Before we proceed to the proof of this theorem, we would like to point out two corollaries. Notice that $P(1 \oplus 1) = X \times S^2$. \par 

\subsection{COROLLARY}\label{cor:2.2.2} \textit{$K(S^2)$ is generated by $[H]$ as a $K$ (point) module, and $[H]$ is subject to the only single relation $([H]-[1])^2 = 0$.} \par 

\subsection{COROLLARY}\label{cor:2.2.3} \textit{If $X$ is any space, and if $\mu: K(X) \otimes K(S^2) \to K(X \times S^2)$ is defined by $\mu(a \otimes b) = (\pi^*_1 a)(\pi^*_2 b)$, where $\pi_1, \pi_2$ are the projections onto the two factors, then $\mu$ is an isomorphism of rings.} \par \hfill

The proof of the theorem will be broken down into a series of lemmas. \par

To begin, we notice that for any $x \in X$, there is a natural embedding $L_x \to P(L \oplus 1)_x$ given by the map $y \to (y, 1)$. This map extends to the one point compactification of $L_x$ and gives us a homeomorphism of the one point compactification of $L_x$ onto $P(L \oplus 1)_x$. If we map $X \to P(L \oplus 1)$ by sending $x$ to the image of the "point at infinity" of the one point compactification of $L_x$, we obtain a section of $P(L \oplus 1)$ which we call the "section at infinity". Similarly, the zero section of $L$ gives us a section of $P(L \oplus 1)$, which we call the zero section of $P(L \oplus 1)$. \par 

We choose a metric on $L$, and we let $S \subset L$ be the unit circle bundle. We write $P^0$ for the part of $L$ consisting  of vectors of length $\leq 1$, and $P^\infty$ for that part of $P(L \oplus 1)$ consisting of the section at infinity, together with all the vectors of length $\geq 1$. We denote the projections $S \to X$, $P^0 \to X$, $P^\infty \to X$ by $\pi$, $\pi_0$, and $\pi_\infty$ respectively \par 

Since $\pi_0$ and $\pi_\infty$ are homotopy equivalences, every bundle on $P^0$ is of the form $\pi^*_0 E^0$ and every bundle on $P^\infty$ is of the form $\pi^*_\infty E^\infty$, where $E^0$ and $E^\infty$ are bundles on $X$. Thus, any bundle $E$ on $P(L \oplus 1)$ is isomorphic to one of the form $(\pi^*_0(E^0), f, \pi^*_\infty(E^\infty))$ where $f \in \ISO(\pi^*(E^0), \pi^*(E^\infty))$ is a clutching function. Moreover, if we insist that the isomorphism
\begin{equation*}
    E \to (\pi^*_0 E^0, f, \pi^*_\infty E^\infty)
\end{equation*}
coincide with the obvious ones over the zero and infinite sections, it follows \textit{that the homotopy class of $f$ is uniquely determined by the isomorphism class of $E$}. This again uses the fact that the $0$-section is a deformation retract of $P^0$ and the $\infty$-section a deformation retract of $P^\infty$. We shall simplify our notation slightly by writing $(E^0, f, E^\infty)$ for $(\pi^*_0(E^0), f, \pi^*_\infty(E^\infty))$ \par 

Our proof will now be devoted to showing that the bundles $E^0$ and $E^\infty$ and the clutching function $f$ can be taken to have a particularly simple form. In the special case that $L$ is trivial, $S$ is just $X \times S^1$, the projection $S \to S^1$ is a complex-valued function on $S$ which we denote by $z$ (here $S^1$ is identified with the complex numbers of unit modulus). This allows us to consider functions on $S$ which are finite Laurent series in $z$ whose coefficients are functions on $X$:
\begin{equation*}
    \sum_{k = -n}^n a_k(x) z^k
\end{equation*}

These finite Laurent series can be used to approximate functions on $S$ in a uniform manner. \par 

If $L$ is not trivial, we have an analogue to finite Laurent series. Here $z$ becomes a section in a bundle rather than a function. Since $\pi^*(L)$ is a subset of $S \times L$, the diagonal map $S \to S \times S \subset S \times L$ gives us a section of $\pi^*(L)$. We denote this section by $z$. Taking tensor products we obtain, for $k \geq 0$, a section $z^k$ of $(\pi^*(L))^k$, and a section $z^{-k}$ of $(\pi^*(L^*))^k$ . We write $L^{-k}$ for $(L^*)^k$. Then, for any $k, k'$, $L^k \otimes L^{k'} \cong L^{k+k'}$. Suppose that $a_k \in \Gamma(L^{-k})$. Then $\pi^*(a_k) \otimes z^k \in \Gamma(\pi^*(1))$, and thus $\pi^*(a_k) \otimes z^k$ is a function on $S$. We write $a_k z^k$ for this function. By a finite Laurent series, we shall understand a sum of functions on $S$ of the form
\begin{equation*}
    \sum_{k = -n}^n a_k z^k
\end{equation*}
where $a_k \in \Gamma(L^{-k})$ for all $k$. \par 

More generally, if $E^0$, $E^\infty$ are two vector bundles on $X$, and $a_k \in \Gamma\Hom(L^k \otimes E^0, E^\infty)$, then if we write $a_k z^k$ for $a_k \otimes z^k$, we see that any finite sum of the form
\begin{equation*}
    f = \sum_{k = -n}^n a_k z^k
\end{equation*}
is an element of $\Gamma\Hom(\pi^*(E^0), \pi^*(E^\infty))$. If $f \in \ISO(\pi^*(E^0), \pi^*(E^\infty))$, we call $f$ a \textit{Laurent clutching function} for $(E^0, E^\infty)$. \par 

The function $z$ is a clutching function for $(1, L)$. Further, $(1, z, L)$ is just the bundle $H^*$ which we defined earlier. To see this, we first recall that $H^*$ was defined as a sub-bundle of $\pi^*(L \oplus 1)$. For each $y \in P(L \oplus 1)$, $H^*$ is a subspace of $(L \oplus 1)_x$, and
\begin{equation*}
    H^*_\infty = L_x \oplus 0, \qquad H^*_0 = 0 \oplus 1_x
\end{equation*}

Thus, the composition
\begin{equation*}
    H^* \to \pi^*(L \oplus 1) \to \pi^*(1)
\end{equation*}
induced by the projection $L \oplus 1 \to 1$ defines an isomorphism:
\begin{equation*}
    f_0: H^* \vert P^0 \to \pi^*_0(1)
\end{equation*}
Likewise, the composition
\begin{equation*}
    H^* \to \pi^*(L \oplus 1) \to \pi^*(L)
\end{equation*}
induced by the projection $L \oplus 1 \to L$ defines an isomorphism:
\begin{equation*}
    f_\infty: H^* \vert P^\infty \to \pi^*_0(L)
\end{equation*}
Hence $f = f_\infty f_0^{-1}: \pi^*(1) \to \pi^*(L)$ is a clutching function for $H^*$. Clearly, if $y \in S_x$, $f(y)$ is the isomorphism whose graph is $H^*_y$. Since $H^*_y$ is the subspace of $L_x \oplus 1_x$ spanned by $y \oplus 1, \quad (y \in S_x \subset L_x, 1 \in \mathbb{C})$, we see that $f$ is exactly our section $z$. Thus
\begin{equation*}
    H^* \cong (1, z, L).
\end{equation*}
Therefore, for any integer $k$,
\begin{equation*}
    H^k \cong (1, z^{-k}, L^{-k}).
\end{equation*} \par 

The next step in our classification of the bundles over $P$ is to show that every clutching function can be taken to be a Laurent clutching function. Suppose that $f \in \Gamma\Hom(\pi^* E^0, \pi^* E^\infty)$ is any section. We define its Fourier coefficients
\begin{equation*}
    a_k \in \Gamma\Hom(L^k \otimes E^0, E^\infty)
\end{equation*}
by
\begin{equation*}
    a_k(x) = \frac{1}{2 \pi i} \int_{S_x} f_x z_x^{-k-1} dz_x .
\end{equation*}
Here $f_x$, $z_x$ denote the restrictions of $f$, $z$ to $S_x$ and $dz_x$ is therefore a differential on $S_x$ with coefficients in $L_x$. Let $S_n$ be the partial sum
\begin{equation*}
    S_n = \sum_{-n}^n a_k z^k
\end{equation*}
and define the Cesaro means
\begin{equation*}
    f_n = \frac{1}{n} \sum_0^{n-1} S_k .
\end{equation*}
Then the proof of Fejér's theorem on the $(\mathbb{C}, 1)$ summability of Fourier series extends immediately to the present more general case and gives

\subsection{LEMMA}\label{2.2.4} \textit{Let $f$ be any clutching function for $(E^0, E^\infty)$ and let $f_n$ be the sequence of Cesaro means of the Fourier series of $f$. Then $f_n$ converges uniformly to $f$. Thus, for all large $n$, $f_n$ is a clutching function for $(E^0, E^\infty)$ and $(E^0, f, E^\infty) \cong (E^0, f_n, E^\infty)$.} \par 

\textbf{Proof:} Since $\ISO(E^0, E^\infty)$ is an open subset of the vector space $\HOM(E^0, E^\infty)$, there exists an $\epsilon > 0$ such that $g \in \HOM(E^0, E^\infty)$ whenever $|f - g| < \epsilon$, where $|\cdot|$ denotes the usual $sup$ norm with respect to fixed metrics in $E^0, E^\infty$. Since the $f_n$ converge uniformly to $f$ we have $|f - f_n| < \epsilon$ for large $n$. Thus, for $0 \leq t \leq 1$, $tf + (1-t)f_n \in \ISO(E^0, E^\infty)$, $f$ and $f_n$ are homotopic in $\ISO(E^0, E^\infty)$, so $(E^0, f, E^\infty) \cong (E^0, f_n, E^\infty)$. \par \hfill

Next, consider a polynomial clutching function; that is, one of the form
\begin{equation*}
    p = \sum_{k = 0}^n a_k z^k
\end{equation*}
Consider the homomorphism
\begin{equation*}
    \mathcal{L}^n(p): \pi^*\left( \sum_{k = 0}^n L^k \otimes E^0 \right) \to \pi^*\left( E^\infty \oplus \sum_{k = 1}^n L^k \otimes E^0 \right)
\end{equation*}
given by the matrix
\begin{equation*}
    \mathcal{L}^n(p) =
    \begin{pmatrix}
        a_0 & a_1 & a_2 & \ldots & a_n \\
        -z  &  1  &     &        &     \\
            & -z  &  1  &        &     \\
            &     &\ddots& \ddots&     \\
            &     &     &   -z   &  1
    \end{pmatrix}
\end{equation*}
It is clear that $\mathcal{L}^n(p)$ is \textit{linear} in $z$. Now, define the sequence $p_r(z)$ inductively by
\begin{equation*}
    p_0 = p, \quad zp_{r+1}(z) = p_r(z) - p_r(0).
\end{equation*}
Then we have the following matrix identity:
\begin{equation*}
    \mathcal{L}^n(p) =
    \begin{pmatrix}
        1   & p_1 & p_2 & \ldots & p_n \\
            &  1  &     &        &     \\
            &     &  1  &        &     \\
            &     &     & \ddots &     \\
            &     &     &        &  1
    \end{pmatrix}
        \begin{pmatrix}
        p   &     &     &        &     \\
            &  1  &     &        &     \\
            &     &  1  &        &     \\
            &     &     & \ddots &     \\
            &     &     &        &  1
    \end{pmatrix}
        \begin{pmatrix}
        1   &     &     &        &     \\
        -z  &  1  &     &        &     \\
            & -z  &  1  &        &     \\
            &     &\ddots& \ddots&     \\
            &     &     &   -z   &  1
    \end{pmatrix}
\end{equation*}
or, more briefly
\begin{equation*}
    \mathcal{L}^n(p) = (1 + N_1)(p \oplus 1)(1 + N_2)
\end{equation*}
where $N_1$ and $N_2$ are nilpotent. If $N$ is nilpotent, $1 + tN$ is nonsingular for $0 \leq t \leq 1$, so we obtain

\subsection{PROPOSITION}\label{pro:2.2.5} \textit{$\mathcal{L}^n(p)$ and $p \oplus 1$ define isomorphic bundles on $P$, i.e.,}
\begin{equation*}
    (E^0, p, E^\infty) \oplus \left( \sum_{k = 1}^n L^k \otimes E^0, 1, \sum_{k = 1}^n L^k \otimes E^0 \right) \cong \left( \sum_{k = 0}^n L^k \otimes E^0, \mathcal{L}^n(p), E^\infty \oplus \sum_{k = 1}^n L^k \otimes E^0 \right)
\end{equation*} \par

\textbf{Remark:} The definition of $\mathcal{L}^n(p)$ is, of course, modelled on the way one passes from an ordinary differential equation of order $n$ to a system of first order equations. \par 

For brevity, we write $\mathcal{L}^n(E^0, p, E^\infty)$ for the bundle
\begin{equation*}
    \left( \sum_{k = 0}^n L^k \otimes E^0, \mathcal{L}^n(p), E^\infty \oplus \sum_{k = 1}^n L^k \otimes E^0 \right)
\end{equation*}

\subsection{LEMMA}\label{lem:2.2.6} \textit{Let p be a polynomial clutching function of degree $\leq n$  for $(E_0, E^\infty)$. Then}
\begin{enumerate}[(i)]
    \item $\mathcal{L}^{n+1}(E^0, p, E^\infty) \cong \mathcal{L}^n(E^0, p, E^\infty) \oplus (L^{n+1} \otimes E^0, 1, L^{n+1} \otimes E^0)$
    \item $\mathcal{L}^{n+1}(L^{-1} \otimes E^0, zp, E^\infty) \cong \mathcal{L}^n(E^0, p, E^\infty) \oplus (L^{-1} \otimes E^0, z, E^0)$
\end{enumerate}

\textbf{Proof:} We have
\begin{equation*}
    \mathcal{L}^{n+1}(p) =
    \begin{pmatrix}
        \mathcal{L}^n(p) &&&&& 0 \\
        0 & 0 & \ldots & & -z & 1
    \end{pmatrix}
\end{equation*}
Multiplying the $z$ on the bottom row by $t$ gives us a homotopy between $\mathcal{L}^{n+1}(p)$ and $\mathcal{L}^n(p) \oplus 1$. This establishes the first part. \par 

Similarly,
\begin{equation*}
    \mathcal{L}^{n+1}(zp) =
    \begin{pmatrix}
        0   & a_0 & a_1 & \ldots & a_n \\
        -z  &  1  &     &        &     \\
            & -z  &  1  &        &     \\
            &     &\ddots& \ddots&     \\
            &     &     &   -z   &  1
    \end{pmatrix}
\end{equation*}
We multiply the $1$ on the second row by $t$ and obtain a homotopy between $\mathcal{L}^{n+1}(zp)$ and $\mathcal{L}^n(p) \oplus (-z)$. Since $-z$ is the composition of $z$ with the map $-1$, and since $-1$ extends to $E^0$, $(L^{-1} \otimes E^0, -z, E^0) \cong (L^{-1} \otimes E^0, z, E^0)$. The second part is therefore proved. \par 

We shall now establish a simple algebraic formula in $K(P)$. We write $[E^0, z, E^\infty]$ for $[(E^0, z, E^\infty)]$.

\subsection{PROPOSITION}\label{pro:2.2.7} \textit{For any polynomial clutching function $p$ for $(E^0, E^\infty)$, we have the identity}
\begin{equation*}
    ([E^0, p, E^\infty] - [E^0, 1, E^0])([L][H] - [1]) = 0.
\end{equation*} \par 

\textbf{Proof:} From the second part of the last lemma, together with the last proposition, we see that
\begin{equation*}
    (L^{-1} \otimes E^0, zp, E^\infty) \oplus \left( \sum_{k = 0}^n L^k \otimes E^0, 1, \sum_{k = 0}^n L^k \otimes E^0  \right)
\end{equation*}
\begin{equation*}
    \cong (E^0, p, E^\infty) \oplus \left( \sum_{k = 1}^n L^k \otimes E^0, 1, \sum_{k = 1}^n L^k \otimes E^0  \right) \oplus (L^{-1} \otimes E^0, z, E^0) .
\end{equation*}
Thus, in $K(P)$,
\begin{equation*}
    [L^{-1} \otimes E^0, zp, E^\infty] \oplus [E^0, 1, E^0] = [E^0, p, E^\infty] \oplus [L^{-1} \otimes E^0, z, E^0]
\end{equation*}
Since $[1, z, L] = [H^{-1}]$,
\begin{equation*}
    [L^{-1}][H^{-1}][E^0, p, E^\infty] \oplus [E^0, 1, E^0] = [E^0, p, E^\infty] \oplus [L^{-1}][H^{-1}][E^0, 1, E^0]
\end{equation*} \par 

In particular, if we put $E^0 = 1, p = z, E^\infty = L$, we obtain the formula
\begin{equation*}
    ([H] - [1])([L][H] - [1]) = 0
\end{equation*}
which is part of our main theorem. \par 

We now turn our attention to linear clutching functions. First, suppose that $T$ is an endomorphism of a finite dimensional vector space $E$, and let $S$ be a circle in the complex plane which does not pass through any eigenvalue of $T$. Then
\begin{equation*}
    Q = \frac{1}{2 \pi i} \int_S (z - T)^{-1} dz
\end{equation*}
is a projection operator in $E$ which commutes with $T$. The decomposition $E = E_+ \oplus E_-$, $E_+ = QE$, $E_- = (1 - Q)E$ is therefore invariant under $T$, so that $T$ can be written as $T = T_+ \oplus T_-$. Then $T_+$ has all of its eigenvalues \textit{inside} $S$, while $T_-$ has all of its eigenvalues \textit{outside} $S$. This is, of course, just the spectral decomposition of $T$ corresponding to the two components of the complement of $S$. \par 

We shall now extend these results to vector bundles, but first we make a remark on notation. So far $z$ and hence $p(z)$ have been sections over $S$. However, they extend in a natural way to sections over the whole of $L$. It will also be convenient to include the $\infty$-section of $P$ in certain statements. Thus, if we assert that $p(z) = az + b$ is an isomorphism outside $S$, we shall take this to include the statement that $a$ is an isomorphism. \par 

\subsection{PROPOSITION}\label{pro:2.2.8} \textit{Let $p$ be a linear clutching
function for $(E^0, E^\infty)$, and define endomorphisms $Q^0, Q^\infty$ of $E^0, E^\infty$ by putting}
\begin{equation*}
    Q^0_x = \frac{1}{2 \pi i} \int_{S_x} p_x^{-1} dp_x \qquad Q^\infty_x = \frac{1}{2 \pi i} \int_{S_x} dp_x p_x^{-1}
\end{equation*}
\textit{Then $Q^0$ and $Q^\infty$ are projection operators, and}
\begin{equation*}
    pQ^0 = Q^\infty p .
\end{equation*}
\textit{Write $E^i_+ = Q^i E^i$, $E^i_- = (1 - Q^i) E^i$, $i = 0, \infty$, so that $E^i = E^i_+ \oplus E^i_-$. Then $p$ is compatible with these decompositions, so that $p = p_+ \oplus p_-$. Moreover, $p_+$ is an isomorphism outside $S$, and $p_-$ is an isomorphism inside $S$.}

\textbf{Proof:} It suffices to verify these statements at each point $x \in X$. In other words, we may assume that $X$ is a point, $L = \mathbb{C}$, and $z$ is just a complex number. Since $p(z)$ is an isomorphism for $|z| = 1$, we can find a real number $\alpha$ with $\alpha > 1$ such that $p(\alpha): E^0 \to E^\infty$ is an isomorphism. For simplicity of computation, we identify $E^0$ with $E^\infty$ by this isomorphism. Next, we consider the conformal transformation
\begin{equation*}
    w = \frac{1 - \alpha z}{z - \alpha}
\end{equation*}
which preserves the unit circle and its inside. Substituting for $z$, we find (since we have taken $p(\alpha) = 1$)
\begin{equation*}
    p(z) = \frac{w - T}{w + \alpha}
\end{equation*}
where $T \in \End(E^0)$. Hence
\begin{align*}
    Q^0 &= \frac{1}{2 \pi i} \int_{|z| = 1} p^{-1} dp \\
    &= \frac{1}{2 \pi i} \int_{|w| = 1} (-(w + \alpha)^{-1} dw + (w - T)^{-1} dw) .\\
    &= \frac{1}{2 \pi i} \int_{|w| = 1} (w - T)^{-1} dw \qquad \text{since } |\alpha| > 1.
\end{align*}
Similarly,
\begin{equation*}
    Q^\infty = \frac{1}{2 \pi i} \int_{|w| = 1} (dw)(w - T)^{-1} = Q^0,
\end{equation*}
so our assertions follow from the corresponding statements concerning a linear transformation $T$.

\subsubsection{COROLLARY}\label{cor:2.2.9} \textit{Let $p$ be as in (\ref{pro:2.2.8}), and write}
\begin{equation*}
    p_+ = a_+ z + b_+, \qquad p_- = a_- z + b_- .
\end{equation*}
\textit{Then, if $p(t) = p_+(t) \oplus p_-(t)$, where}
\begin{equation*}
    p_+(t) = a_+ z + tb_+, \qquad p_-(t) = ta_- z + b_-, \qquad 0 \leq t \leq 1,
\end{equation*}
\textit{we obtain a homotopy of linear clutching functions connecting $p$ with $a_+ z \oplus b_-$. Thus}
\begin{equation*}
    (E^0, p, E^\infty) \cong (E^0_+, z, L \otimes E^0_+) \oplus (E^0_-, 1, E^0_-) .
\end{equation*}

\textbf{Proof:} The last part of the last lemma implies that $p_+(t)$ and $p_-(t)$ are isomorphisms on $S$ for $0 \leq t \leq 1$. Thus, $p(t)$ is a clutching function for $0 \leq t \leq 1$. Thus,
\begin{align*}
    (E^0, p, E^\infty) &\cong (E^0, p(1), E^\infty) \\
    &\cong (E^0_+, a_+ z, E^\infty_+) \oplus (E^0_-, b_-, E^\infty_-) .
\end{align*}
Since $a_+: L \otimes E^0_+ \to E^\infty_+$, $b_-: E^0_- \to E^\infty_-$ are necessarily isomorphisms, we see that
\begin{align*}
    (E^0_+, a_+ z, E^\infty_+) &\cong (E^0_+, z, L \otimes E^\infty_+) \\
    (E^0_-, b_-, E^\infty_-) &\cong (E^0_-, 1, E^0_-) .
\end{align*}
Again, consider a polynomial clutching function $p$ of degree $\leq n$. Then $\mathcal{L}^n(p)$ is a linear clutching function for $(V^0, V^\infty)$ where
\begin{equation*}
    V^0 = \sum_{k=0}^\infty L^k \otimes E^0, \qquad V^\infty = E^\infty \oplus \sum_{k=1}^\infty L^k \otimes E^0
\end{equation*}
Hence, it defines a decomposition
\begin{equation*}
    V^0 = V^0_+ \oplus V^0_-
\end{equation*}
as above. To express the dependence of $V^0_+$ on $p$ and $n$, we write
\begin{equation*}
    V^0_+ = V_n(E^0, p, E^\infty) .
\end{equation*}
Note that this is a vector bundle on $X$. If $p_t$ is a homotopy of polynomial clutching functions of degree $\leq n$, it follows by constructing $V_n$ over $X \times I$ that
\begin{equation*}
    V_n(E^0, p_0, E^\infty) \cong V_n(E^0, p_1, E^\infty) .
\end{equation*}
Hence, from the homotopies used in proving the two parts of (\ref{lem:2.2.6}), we obtain
\begin{align*}
    V_{n+1}(E^0, p, E^\infty) &\cong V_n(E^0, p, E^\infty), \\
    V_{n+1}(L^{-1} \otimes E^0, zp, E^\infty) &\cong V_n(E^0, p, E^\infty) \oplus (L^{-1} \otimes E^0)
\end{align*}
or, equivalently
\begin{equation*}
    V_{n+1}(E^0, zp, L \otimes E^\infty) \cong L \otimes V_n(E^0, p, E^\infty) \oplus E^0 .
\end{equation*} \par 
Combining this with the above corollary and (\ref{pro:2.2.5}), we obtain the following formula in $K(P)$:
\begin{align*}
    [E^0, p, E^\infty] + \left\{\sum_{k=1}^n [L^k \otimes E^0] \right\}[1] =& [V_n(E^0, p, E^\infty)][H^{-1}] \\
    &+ \left\{\sum_{k=0}^n [L^k \otimes E^0] - [V_n(E^0, p, E^\infty)] \right\}[1]
\end{align*}
and hence the formula
\begin{equation*}
    [E^0, p, E^\infty] = [V_n(E^0, p, E^\infty)]([H^{-1}] - [1]) + [E^0][1] .
\end{equation*}
This shows that $[V^+_p] \in K(X)$ completely determines $[E^0, p, E^\infty] \in K(P)$. We can now prove our theorem. \par

Let $t$ be an indeterminant over the ring $K(X)$. Then the map $t \to [H]$ induces a $K(X)$-algebra homomorphism (since $([H] - [1])([L][H] - [1]) = 0$)
\begin{equation*}
    \mu: K(X)[t] / ((t-1)([L]t-1)) \to K(P) .
\end{equation*}
To prove that $\mu$ is an isomorphism, we explicitly construct an inverse. \par 

First, suppose that $f$ is a clutching function for $(E^0, E^\infty)$. Let $f_n$ be the sequence of Cesaro means of its Fourier series, and put $p_n = z^n f_n$. Then, if $n$ is sufficiently large, $p_n$ is a polynomial clutching function (of degree $\leq 2n$) for $(E^0, L^n \otimes E^\infty)$. We define
\begin{equation*}
    \nu_n(f) = [V_{2n}(E^0, p_n, L^n \otimes E^\infty)](t^{n-1} - t^n) + [E^0]t^n .
\end{equation*}
Now, for sufficiently large $n$, the linear segment joining $p_{n+l}$ and $zp_n$ provides a homotopy of polynomial clutching functions of degree $\leq 2(n + 1)$. Hence, by the formulae following (\ref{cor:2.2.9}),
\begin{align*}
    V_{2n+2}(E^0, p_{n+1}, L^{n+1} \otimes E^\infty) &\cong V_{2n+2}(E^0, zp_n, L^{n+1} \otimes E^\infty) \\
    &\cong V_{2n+1}(E^0, zp_n, L^{n+1} \otimes E^\infty) \\
    &\cong L \otimes V_{2n}(E^0, p_n, L^n \otimes E^\infty) \oplus E^0 .
\end{align*}
Hence
\begin{align*}
    \nu_{n+1}(f) &= \{[L] [V_{2n}(E^0, p_n, L^n \otimes E^\infty)] + [E^0] \}(t^n - t^{n+1}) + [E^0]t^{n+1} \\
    &= \nu_n(f)
\end{align*}
since $(t - l)([L]t - 1) = 0$. Thus, $\nu_n(f)$ is independent of $n$ if $n$ is sufficiently large, and thus depends only on $f$. We write it as $\nu(f)$. If $g$ is sufficiently close to $f$, and $n$ is sufficiently large, the linear segment joining $f_n$ and $g_n$ provides a homotopy of polynomial clutching functions of degree $\leq 2n$, and hence $\nu(f) = \nu_n(f) = \nu_n(g) = \nu(g)$. Thus, $\nu(f)$ is a locally constant function of $f$, and hence depends only on the homotopy class of $f$. However, if $E$ is any bundle on $P$, and $f$ a clutching function defining $E$, we define $\nu(E) = \nu(f)$, and $\nu(E)$ will be well defined and depend only on the isomorphism class of $E$. Since $\nu(E)$ is clearly additive for $+$, it induces a group homomorphism
\begin{equation*}
    \nu: K(P) \to K(X)[t] / ((t-1)([L]t-1)) .
\end{equation*}
From our definition, it is clear that this is a $K(X)$-module homomorphism. \par 

First, we check that $\mu \nu$ is the identity. With our notation as above,
\begin{align*}
    \mu\nu(E) &= \mu\{[V_{2n}(E^0, p_n, L^n \otimes E^\infty)](t^{n-1} - t^n) + [E^0]t^n \} \\
    &= [V_{2n}(E^0, p_n, L^n \otimes E^\infty)]([H]^{n-1} - [H^n]) + [E^0][H]^n \\
    &= [E^0, p_n, L^n \otimes E^\infty] [H]^n \\
    &= [E^0, f_n, E^\infty] \\
    &= [E^0, f, E^\infty] \\
    &= E .
\end{align*}
Since $K(P)$ is additively generated by elements of the form $[E]$, this proves that $\mu\nu$ is the identity. \par 

Finally, we show that $\nu\mu$ is the identity. Since $\nu\mu$ is a homomorphism of $K(X)$-modules, it suffices to show that $\nu\mu(t^n) = t^n$ for all $n \geq 0$. However,
\begin{align*}
    \nu\mu(t^n) &= \nu(H^n) \\
    &= \nu[1, z^{-n}, L^{-n}] \\
    &= [V_{2n}(1, 1, 1)](t^{n-1} - t^n) + [1]t^n \\
    &= t^n, \qquad \text{since } V_{2n}(1, 1, 1) = 0 .
\end{align*}

\newpage

%%%%%%%%%%%%%%%%%%%%%%%%%%%%%%%%%%%%%%%

\section{$K_G(X).$} Suppose that $G$ is a finite group and that $X$ is a $G$-space. Let $\Vect_G(X)$ denote the set of isomorphism classes of $G$-vector bundles over $X$. This is an abelian semi-group under $\oplus$. We form the associated abelian group and denote it by $K_G(X)$. If $G = 1$ is the trivial group then $K_G(X) = K(X)$. If on the other hand $X$ is a point then $K_G(X) \cong R(G)$ the character ring of $G$. \par 

If $E$ is a $G$-vector bundle over $X$ then $P(E)$ is a $G$-space. If $E = L \oplus 1$ when $L$ is a $G$-bundle then the zero and infinite sections $X \to P(E)$ are both $G$-sections. Also the bundle $H$ over $P(E)$ is a $G$-line bundle. If we now examine the proof of the periodicity theorem which we have just given we see that we could have assumed a $G$-action on everything. Thus we get the periodicity theorem for $K_G$: \par 

\subsection{THEOREM}\label{the:2.3.1} \textit{If $X$ is a $G$-space, and if $L$ is $G$-line bundle over $X$, the map $t \to [H]$ induces an isomorphism of $K_G(X)$-modules:}

\begin{equation*}
    K_G(X)[t]/(t[L]-1)(t-1) \to K_G(P(L \oplus 1))
\end{equation*}

\newpage

%%%%%%%%%%%%%%%%%%%%%%%%%%%%%%%%%%%%%%%

\section{Cohomology theory properties of $K$.} We next define $K(X, Y)$ for a compact pair $(X, Y)$. We shall then be able to establish, in a purely formal fashion, certain properties of $K$. Since the proofs are formal, the theorems are equally valid for any "cohomology theory" satisfying certain axioms. We leave this formalization to the reader. \par 

Let $\mathcal{C}$ denote the category of compact spaces, $\mathcal{C}^*$ the category of compact spaces with distinguished basepoint, and $\mathcal{C}^2$ the category of compact pairs. We define functors:

\begin{center}
\begin{tikzcd}
\mathcal{C}^2 \arrow[r] & \mathcal{C}^* \\
\mathcal{C} \arrow[r]   & \mathcal{C}^2
\end{tikzcd}
\end{center}

by sending a pair $(X, Y)$ to $X/Y$ with basepoint $Y/Y$ (if $Y \neq \emptyset$, the empty set, $X/Y$ is understood to be the disjoint union of $X$ with a point.) We send a space $X$ to the pair $(X, \emptyset)$. The composite $\mathcal{C} \to \mathcal{C}^*$ is given by $X \to X^*$, where $X^*$ denotes $X/\emptyset$. \par

If $X$ is in $\mathcal{C}^*$, we define $\tilde{K}(X)$ to be the kernel of the map $i^*: K(X) \to K(x_O)$ where $i: x_0 \to X$ is the inclusion of the base-point. If $c: X \to x_0$ is the collapsing map then $c^*$ induces a splitting $K(X) \cong \tilde{K}(X) \oplus K(x_0)$. This splitting is clearly natural for maps in $\mathcal{C}^*$. Thus $\tilde{K}$ is a functor on $\mathcal{C}^*$. Also, it is clear that $K(X) \cong \tilde{K}(X^*)$. We define $K(X, Y)$ by $K(X, Y) = \tilde{K}(X/Y)$. In particular $K(X, \emptyset) = K(X)$. Since $\tilde{K}$ is a functor on $\mathcal{C}^*$ it follows that $K(X, Y)$ is a contravariant functor of $(X, Y)$ in $\mathcal{C}^2$. \par 

We now introduce the "smash product" operation in $\mathcal{C}^*$. If $X, Y \in \mathcal{C}$ we put

\begin{equation*}
    X \wedge Y = X \times Y / X \vee Y
\end{equation*}

where $X \vee Y = X \times y_0 \cup x_0 \times Y$, $x_0, y_0$ being the base-points of $X, Y$ respectively. For any three spaces $X, Y, Z \in \mathcal{C}^*$ we have a natural homeomorphism

\begin{equation*}
    X \wedge (Y \wedge Z) \approx (X \wedge Y) \wedge Z
\end{equation*}

and we shall identify these spaces by the homeomorphism. \par 

Let $I$ denote the unit interval $[0, 1]$ and let $\partial I$ = $\{0\} \cup \{1\}$ be its boundary. We take $I/\partial I \in \mathcal{C}^*$ as our standard model of the circle $S^1$. Similarly if $I^n$ denotes the unit cube in $\mathbb{R}^n$ we take $I^n/\partial I^n$ as our model of the $n$-sphere $S^n$. Then we have a natural homeomorphism

\begin{equation*}
S^n \approx S^1 \wedge S^1 \wedge \ldots \wedge S^1 \qquad \text{($n$ factors)}
\end{equation*}

For $X \in \mathcal{C}^*$ the space $S^1 \wedge X  \in \mathcal{C}^*$ is called the \textit{reduced suspension} of $X$, and often written as $SX$. The $n$-th iterated suspension $SS \ldots SX$ ($n$ times) is naturally homeomorphic to $S^n \wedge X$ and is written briefly as $S^n X$.

\begin{align*}
    \tilde{K}^{-n}(X) &= \tilde{K}(S^n X) & \textit{ for } X \in \mathcal{C}^* \\
    K^{-n}(X, Y) &= \tilde{K}^{-n}(X/Y) = \tilde{K}(S^n(X/Y)) & \textit{ for } (X, Y) \in \mathcal{C}^2 \\
    K^{-n}(X) &= K^{-n}(X, \emptyset) = \tilde{K}(S^n(X^*)) & \textit{ for } X \in \mathcal{C}
\end{align*}

It is clear that all these are contravariant functors on the appropriate categories. \par

Before proceeding further we define the \textit{cone on} $X$ by

\begin{equation*}
    CX = I \times X/\{0\} \times X
\end{equation*}

Thus $C$ is a functor $C: \mathcal{C} \to \mathcal{C}^*$. We identify $X$ with the subspace $\{l\} \times X$ of $CX$. The space $CX/X = I \times X/\partial I \times X$ is called the \textit{unreduced suspension} of $X$. Note that this is a functor $\mathcal{C} \to \mathcal{C}^*$ whereas the reduced suspension $S$ is a functor $\mathcal{C}^* \to \mathcal{C}^*$. If $X \in \mathcal{C}^*$ with base-point $x_0$ then we have a natural inclusion map

\begin{equation*}
    I \approx Cx_0 / x_0 \to CX/X
\end{equation*}

and the quotient space obtained by collapsing $I$ in $CX/X$ is just $SX$. Thus by (\ref{lem:1.4.8}) $p: CX/X \to SX$ induces an isomorphism $K(SX) \cong K(CX/X)$ and hence also an isomorphism $\tilde{K}(SX) \approx K(CX,X)$. Thus the use of $SX$ for both the reduced and unreduced suspensions leads to no problems. \par

If $(X, Y) \in \mathcal{C}^2$ we define $X \cup CY$ to be the space obtained from $X$ and $CY$ by identifying the subspaces $Y \subset X$ and $\{1\} \times Y \subset CY$. Taking the base-point of $CY$ as base-point of $X \cup CY$ we have

\begin{equation*}
    X \cup CY \in \mathcal{C}^*
\end{equation*}.

We note that $X$ is a subspace of $X \cup CY$ and that there is a natural homeomorphism

\begin{equation*}
    X \cup CY/X \approx CY/Y
\end{equation*}.

Thus if $Y \in \mathcal{C}^*$,

\begin{align*}
    K(X \cup CY, X) &\cong K(CY, Y) \\
    &\cong \tilde{K}(SY) \\
    &= \tilde{K}^{-1}(Y)
\end{align*}

Now we begin with a simple lemma
