\chapter{\scshape $K$-Theory}

\section{Definitions.}\label{sec:2.1} If $X$ is any space, the set $\Vect(X)$ has the structure of an abelian semigroup, where the additive structure is defined by direct sum. If $A$ is any abelian semigroup, we can associate to $A$ an abelian group $K(A)$ with the following property: there is a semigroup homomorphism $\alpha: A \to K(A)$ such that if $G$ is any group, $\gamma: A \to G$ any semigroup homomorphism there is a unique homomorphism $\chi: K(A) \to G$ such that $\gamma = \chi \alpha$. If such a $K(A)$ exists, it must be unique. \par

The group $K(A)$ is defined in the usual fashion. Let $F(A)$ be the free abelian group generated by the elements of $A$, let $E(A)$ be the subgroup of $F(A)$ generated by those elements of the form $a+a' - (a \oplus a')$ where $\oplus$ is the addition in $A$, $a, a' \in A$. Then $K(A) = F(A)/E(A)$ has the universal property described above, with $\alpha: A \to K(A)$ being the obvious map. \par

A slightly different construction of $K(A)$ which is sometimes convenient is the following. Let $\Delta: A \to A \times A$ be the diagonal homomorphism of semi-groups, and let $K(A)$ denote the set of cosets of $\Delta(a)$ in  $A \times A$. It is, a quotient semi-group, but the interchange of factors in $A \times A$ induces an inverse in $K(A)$ so that $K(A)$ is a group. We then define $\alpha_A: A \to K(A)$ to be the composition of $a \to (a, 0)$ with the natural projection $A \times A \to K(A)$ (we assume $A$ has a zero for simplicity). The pair $(K(A), \alpha_A)$ is a functor of $A$ so that if $\gamma: A \to B$ is a semi-group homomorphism we have a commutative diagram
\begin{center}
\begin{tikzcd}
A \arrow[rr, "\alpha_A"] \arrow[dd, "\gamma"'] &  & K(A) \arrow[dd, "K(\gamma)"] \\
                                               &  &                              \\
B \arrow[rr, "\alpha_B"]                       &  & K(B)                        
\end{tikzcd}
\end{center}
If $B$ is a group $\alpha_B$ is an isomorphism. That shows $K(A)$ has the required universal property. \par 

If $A$ is also a semi-ring (that is, $A$ possesses multiplication which is distributive over the addition of $A$) then $K(A)$ is clearly a ring. \par \hfill

If $X$ is a space, we write $K(X)$ for the ring $K(\Vect(X))$. No confusion should result from this notation. If $E \in \Vect(X)$, we shall write $[E]$ for the image of $E$ in $K(X)$. Eventually, to avoid excessive notation, we may simply write $E$ instead of $[E]$ when there is no danger of confusion. \par 

Using our second construction of $K$ it follows that, if $X$ is a space, every element of $K(X)$ is of the form $[E] - [F]$, where $E, F$ are bundles over $X$. Let $G$ be a bundle such that $F \oplus G$ is trivial. We write $\underline{n}$ for the trivial bundle of dimension $n$ . Let $F \oplus G = \underline{n}$. Then $[E] - [F] = [E] + [G] - ([F] + [G]) = [E \oplus G] - [\underline{n}]$. Thus, every element of $K(X)$ is of the form $[H] -  [\underline{n}]$. \par 

Suppose that $E, F$ are such that $[E] = [F]$, then again from our second construction of $K$ it follows that there is a bundle $G$ such that $E \oplus G \cong F \oplus G$. Let $G'$ be a bundle such that $G \oplus G' \cong \underline{n}$. Then $E \oplus G \oplus G' \cong F \oplus G \oplus G'$, so $E \oplus \underline{n} \cong F \oplus \underline{n}$. If two bundles become equivalent when a suitable trivial bundle is added to each of them, the bundles are said to be \textit{stably equivalent}. Thus, $[E] = [F]$ if and only if $E$ and $F$ are stably equivalent. \par 

Suppose $f: X \to Y$ is a continuous map. Then $f^*: \Vect(Y) \to \Vect(X)$ induces a ring homomorphism $f^*: K(Y) \to K(X)$. By (\ref{lem:1.4.3}) this homomorphism depends only on the homotopy class of $f$. \newpage

%%%%%%%%%%%%%%%%%%%%%%%%%%%%%%%%%%%%%%%%%%%%

\section{The periodicity theorem.}\label{sec:2.2} The fundamental theorem for $K$-theory is the periodicity theorem. In its simplest form, it states that for any $X$, there is an isomorphism between $K(X) \otimes K(S^2)$ and $K(X \times S^2)$. This is a special case of a more general theorem which we shall prove. \par 

If $E$ is a vector bundle over a space $X$, and if $E_0 = E - X$ where $X$ is considered to lie in $E$ as the zero section, the non-zero complex numbers act on $E_0$ as a group of fiber preserving automorphisms. Let $P(E)$ be the quotient space obtained from $E_0$ by dividing by the action of the complex number. $P(E)$ is called the projective bundle associated to $E$. If $p: P(E) \to X$ is the projection map, $p^{-1}(x)$ is a complex projective space for all $x \in X$. If $V$ is a vector space, and $W$ is a vector space of dimension one, $V$ and $V \otimes W$ are isomorphic, but not naturally isomorphic. For any non-zero element $\omega \in W$ the map $\nu \to \nu \otimes \omega$ defines an isomorphism between $V$ and $V \otimes W$, and thus defines an isomorphism $P(\omega): P(V) \to P(V \otimes W)$. However, if $\omega'$ is any other non-zero element of $W$, $\omega' = \lambda \omega$ for some non-zero complex number $\lambda$. Thus $P(\omega) = P(\omega')$, so the isomorphism between $P(V)$ and $P(V \otimes W)$ is natural. Thus, if $E$ is any vector bundle, and $L$ is a line bundle, there is a natural isomorphism $P(E) \cong P(E \otimes L)$. \par 

If $E$ is a vector bundle over $X$ then each point $a \in P(E)_x \to P(E_x)$ represents a one-dimensional subspace $H^*_x \subset E_x$. The union of all these defines a subspace $H^* \subset p^*E$, where $p: P(E) \to X$ is the projection. It is easy to check that $H^*$ is a sub-bundle of $p^*E$. In fact, the problem being local we may assume $E$ is a product and then we are reduced to a special case of the Grassmannian already discussed in \cref{sec:1.4}. We have denoted our line-bundle by $H^*$ because we want its dual $H$ (the choice of convention here is dictated by algebro-geometric considerations which we do not discuss here). \par 

We can now state the periodicity theorem. \par 

\subsection{THEOREM}\label{the:2.2.1} \textit{Let $L$ be a line bundle over $X$. Then, as a $K(X)$-algebra $K(P(L \oplus 1))$ is generated by $[H]$, and is subject to the single relation \linebreak $([H] - 1)([L][H] - [1]) = 0$.} \par 

Before we proceed to the proof of this theorem, we would like to point out two corollaries. Notice that $P(1 \oplus 1) = X \times S^2$. \par 

\subsection{COROLLARY}\label{cor:2.2.2} \textit{$K(S^2)$ is generated by $[H]$ as a $K$ (point) module, and $[H]$ is subject to the only single relation $([H]-[1])^2 = 0$.} \par 

\subsection{COROLLARY}\label{cor:2.2.3} \textit{If $X$ is any space, and if $\mu: K(X) \otimes K(S^2) \to K(X \times S^2)$ is defined by $\mu(a \otimes b) = (\pi^*_1 a)(\pi^*_2 b)$, where $\pi_1, \pi_2$ are the projections onto the two factors, then $\mu$ is an isomorphism of rings.} \par \hfill

The proof of the theorem will be broken down into a series of lemmas. \par

To begin, we notice that for any $x \in X$, there is a natural embedding $L_x \to P(L \oplus 1)_x$ given by the map $y \to (y, 1)$. This map extends to the one point compactification of $L_x$ and gives us a homeomorphism of the one point compactification of $L_x$ onto $P(L \oplus 1)_x$. If we map $X \to P(L \oplus 1)$ by sending $x$ to the image of the "point at infinity" of the one point compactification of $L_x$, we obtain a section of $P(L \oplus 1)$ which we call the "section at infinity". Similarly, the zero section of $L$ gives us a section of $P(L \oplus 1)$, which we call the zero section of $P(L \oplus 1)$. \par 

We choose a metric on $L$, and we let $S \subset L$ be the unit circle bundle. We write $P^0$ for the part of $L$ consisting  of vectors of length $\leq 1$, and $P^\infty$ for that part of $P(L \oplus 1)$ consisting of the section at infinity, together with all the vectors of length $\leq 1$. We denote the projections $S \to X$, $P^0 \to X$, $P^\infty \to X$ by $\pi$, $\pi_0$, and $\pi_\infty$ respectively \par 

Since $\pi_0$ and $\pi_\infty$ are homotopy equivalences, every hundle on $P^0$ is of the form $\pi^*_0 E^0$ and every bundle on $P^\infty$ is of the form $\pi^*_\infty E^\infty$, where $E^0$ and $E^\infty$ are bundles on $X$. Thus, any bundle $E$ on $P(L \oplus 1)$ is isomorphic to one of the form $(\pi^*_0(E^0), f, \pi^*_\infty(E^\infty))$ where $f \in \ISO(\pi^*(E^0), \pi^*(E^\infty))$ is a clutching function. Moreover, if we insist that the isomorphism
\begin{equation*}
    E \to (\pi^*_0 E^0, f, \pi^*_\infty E^\infty)
\end{equation*}
coincide with the obvious ones over the zero and infinite sections, it follows \textit{that the homotopy class of $f$ is uniquely determined by the isomorphism class of $E$}. This again uses the fact that the $0$-section is a deformation retract of $P^0$ and the $\infty$-section a deformation retract of $P^\infty$. We shall simplify our notation slightly by writing $(E^0, f, E^\infty)$ for $(\pi^*_0(E^0), f, \pi^*_\infty(E^\infty))$ \par 

Our proof will now be devoted to showing that the bundles $E^0$ and $E^\infty$ and the clutching function $f$ can be taken to have a particularly simple form. In the special case that $L$ is trivial, $S$ is just $X \times S^1$, the projection $S \to S^1$ is a complex-valued function on $S$ which we denote by $z$ (here $S^1$ is identified with the complex numbers of unit modulus). This allows us to consider functions on $S$ which are finite Laurent series in $z$ whose coefficients are functions on $X$:
\begin{equation*}
    \sum_{k = -n}^n a_k(x) z^k
\end{equation*}

These finite Laurent series can be used to approximate functions on $S$ in a uniform manner. \par 

If $L$ is not trivial, we have an analogue to finite Laurent series. Here $z$ becomes a section in a bundle rather than a function. Since $\pi^*(L)$ is a subset of $S \times L$, the diagonal map $S \to S \times S \subset S \times L$ gives us a section of $\pi^*(L)$. We denote this section by $z$. Taking tensor products we obtain, for $k \geq 0$, a section $z^k$ of $(\pi^*(L))^k$, and a section $z^{-k}$ of $(\pi^*(L))^k$ . We write $L^{-k}$ for $(L^*)^k$. Then, for any $k, k'$, $L^k \otimes L^{k'} \cong L^{k+k'}$. Suppose that $a_k \in (L^{-k})$. Then $\pi^*(a_k) \otimes z^k \in \Gamma(\pi^*(1))$, and thus $\pi^*(a_k) \otimes z^k$ is a function on $S$. We write $a_k z^k$ for this function. By a finite Laurent series, we shall understand a sum of functions on $S$ of the form
\begin{equation*}
    \sum_{k = -n}^n a_k(x) z^k
\end{equation*}
where $a_k \in \Gamma(L^{-k})$ for all $k$. \par 

More generally, if $E^0$, $E^\infty$ are two vector bundles on $X$, and $a_k \in \Gamma\Hom(L^k \otimes E^0, E^\infty)$, then if we write $a_k z^k$ for $a_k \otimes z^k$, we see that any finite sum of the form
\begin{equation*}
    f = \sum_{k = -n}^n a_k(x) z^k
\end{equation*}
is an element of $\Gamma(\pi^*(E^0), \pi^*(E^\infty))$. If $f \in \ISO(\pi^*(E^0), \pi^*(E^\infty))$, we call $f$ a \textit{Laurent clutching function} for $(E^0, E^\infty)$. \par 

The function $z$ is a clutching function for $(1, L)$. Further, $(1, z, L)$ is just the bundle $H^*$ which we defined earlier. To see this, we first recall that $H^*$ was defined as a sub-bundle of $\pi^*(L \oplus 1)$. For each $y \in P(L \oplus 1)$, $H^*$ is a subspace of $(L \oplus 1)_x$, and
\begin{equation*}
    H^*_\infty = L_x \oplus 0, \qquad H^*_0 = 0 \oplus 1_x
\end{equation*}

Thus, the composition
\begin{equation*}
    H^* \to \pi^*(L \oplus 1) \to \pi^*(1)
\end{equation*}
induced by the projection $L \oplus 1 \to 1$ defines an isomorphism:
\begin{equation*}
    f_0: H^* \vert P^0 \to \pi^*_0(1)
\end{equation*}
Likewise, the composition
\begin{equation*}
    H^* \to \pi^*(L \oplus 1) \to \pi^*(L)
\end{equation*}
induced by the projection $L \oplus 1 \to L$ defines an isomorphism:
\begin{equation*}
    f_\infty: H^* \vert P^\infty \to \pi^*_0(L)
\end{equation*}
Hence $f = f_\infty f_0^{-1}: \pi^*(1) \to \pi^*(L)$ is a clutching function for $H^*$. Clearly, if $y \in S_x$, $f(y)$ is the isomorphism whose graph is $H^*_y$. Since $H^*_y$ is the subspace of $L_x \oplus 1_x$ spanned by $y \oplus 1, \quad (y \in S_x \subset L_x, 1 \in \mathbb{C})$, we see that $f$ is exactly our section $z$. Thus
\begin{equation*}
    H^* \cong (1, z, L).
\end{equation*}
Therefore, for any integer $k$,
\begin{equation*}
    H^k \cong (1, z^{-k}, L^{-k}).
\end{equation*} \par 

The next step in our classification of the bundles over $P$ is to show that every clutching function can be taken to be a Laurent clutching function. Suppose that $f \in \Gamma\Hom(\pi^* E^0, \pi^* E^\infty)$ is any section. We define its Fourier coefficients
\begin{equation*}
    a_k \in \Gamma\Hom(L^k \otimes E^0, E^\infty)
\end{equation*}
by
\begin{equation*}
    a_k(x) = \frac{1}{2 \pi i} \int_{S_x} f_x z_x^{-k-1} dz_x .
\end{equation*}
Here $f_x$, $z_x$ denote the restrictions of $f$, $z$ to $S_x$ and $dz_x$ is therefore a differential on $S_x$ with coefficients in $L_x$. Let $S_n$ be the partial sum
\begin{equation*}
    S_n = \sum_{-n}^n a_k z^k
\end{equation*}
and define the Cesaro means
\begin{equation*}
    f_n = \frac{1}{n} \sum_0^{n-1} S_k .
\end{equation*}
Then the proof of Fejer's theorem on the $(\mathbb{C}, 1)$ summability of Fourier series extends immediately to the present more general case and gives

\subsection{LEMMA}\label{2.2.4}
